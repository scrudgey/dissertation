\chapter{Quenching timescale}\label{chap-3}
    
Galaxies form a bimodal distribution in rest-frame color at $z < 2$ \citep{Strateva:2001aa,Baldry:2004aa,Bell:2004aa,Williams:2009tt}, meaning galaxies can be broadly categorized as either actively star-forming spirals (the ``blue cloud"), or quiescent ellipticals and lenticulars (the ``red-sequence").
%(Cassata et al. 2008; Brammer et al. 2009).
Although these populations are roughly equivalent in total stellar mass at $z\sim1$, the quiescent galaxy population has nearly doubled in stellar mass, stellar mass density, and number density over the past $\sim7$ Gyr \citep{Arnouts:2007aa,Bell:2004aa,Borch:2006aa,Bundy:2006aa,Brown:2007aa,Faber:2007aa}.

% \mynote{at a given stellar mass?}
Meanwhile, a variety of studies at intermediate redshift show that galaxy properties correlate with local environment \citep{Cooper:2006aa,Cooper:2007aa,Quadri:2007aa,Patel:2009aa}, such that groups and clusters contain more quiescent than active galaxies at a given stellar mass \citep{George:2011aa,Muzzin:2012dw,Presotto:2012aa,Tanaka:2012aa,Nantais:2017aa}.
Moreover, with increasing cluster-centric radius (and thus decreasing time since infall), observations find a relative reduction in the number of quiescent systems \cite[e.g.][]{Presotto:2012aa}.
Together, these results suggest that dense environments shut off (or ``quench") star formation in galaxies —-- a process typically termed ``environmental quenching" \citep{Peng:2010aa}.

Environment has been studied extensively as a driver of galaxy evolution \citep[for a review see][]{Blanton:2009aa}, but the physical mechanism or mechanisms responsible for quenching have yet to be identified, although several candidates have been proposed.
Whatever the underlying cause, it must disrupt the process by which a galaxy converts cold gas into stars.
One possibility is that this gas is directly removed from a galaxy by ram-pressure stripping as it falls at high speed into the hot intra-cluster medium (ICM) of a cluster environment \citep{Gunn:1972aa}.
A galaxy's star formation might also be disrupted by gas heating and morphological disruption caused by gravitational interaction with other satellite galaxies within the cluster \citep[``harassment", see][]{Moore:1996aa}.

As a galaxy forms stars, its cold gas reservoir is replenished by cooling flows from its surrounding hot gas halo \citep{Bauermeister:2010aa}.
``Strangulation" refers to the removal of this hot gas halo by the cluster environment \citep{Larson:1980aa,Merritt:1983aa,Byrd:1990aa}.
Following the interruption of cooling flows, a galaxy would quench as star formation exhausts the remaining cold gas reservoir over a molecular gas depletion time, $t_{\mathrm{depl}}(\mathrm{H_{mol}})$.
If the hot gas halo is not stripped, the role of the environment may be simply to prevent the accretion of fresh gas onto the galaxy, which would then quench over a longer total gas depletion time $t_{\mathrm{depl}}(\mathrm{H_I}+\mathrm{H_{mol}})$.

It is also possible that gas-dynamical feedback and outflows play a central role in removing the gas from galaxy halos \citep{McGee:2014aa,Balogh:2016aa}.
In this ``overconsumption" scenario, the depletion of gas is augmented by outflows produced by star formation, either directly through radiation pressure or from subsequent supernovae \citet{McGee:2014aa}.
Quenching then proceeds over an accelerated gas depletion timescale that is inversely proportional to the star formation rate (SFR).

These processes would quench galaxies over different timescales.
Measuring the quenching timescale, $t_Q$, is therefore one approach to narrowing down the mechanisms responsible for quenching.
This strategy benefits from the fact that the redshift dependence of these timescales is such that differences will become more apparent at high redshift.
\citet{Nantais:2016aa,Nantais:2017aa} finds a strong evolution in environmental quenching efficiency between $0.9 < z < 1.6$, while over a similar redshift range \citet{Cerulo:2016aa,Cerulo:2017aa} reports an accelerated build-up of the red-sequence in clusters, and \citet{Balogh:2016aa} indicates a change in the environmental quenching mechanism at $z\sim1$.
Measuring $t_Q$ at high redshift is clearly essential to our understanding of environmental quenching.

In this work, we will measure the quenching timescale in a sample of four galaxy clusters at $z\sim1.6$, a higher redshift than has been studied previously.
We will use our results, together with studies drawn from the literature, to investigate the redshift evolution of $t_Q$ and compare with the timescales predicted by various quenching mechanisms.

The structure of this paper is as follows:
Our data set is described in Section \ref{sec-data}.
In Section \ref{sec-analysis}, we summarize our toy model of environmental quenching, which is described in detail in Appendix \ref{sec-math}.
In Section \ref{sec-results} we report the results of our technique, which we discuss in Section \ref{sec-discussion}.
In Section \ref{sec-conclusion} we summarize our conclusions.

In this work we will assume a standard $\Lambda$CDM cosmology with $H_0 = 70 \mathrm{\ km \cdot s^{-1} \cdot Mpc^{-1}}, \Omega_M = 0.3, \mathrm{\ and\ } \Omega_\Lambda = 0.7$, and a Chabrier IMF \citep{chabrierimf} throughout.
Our magnitudes are reported in the AB system \citep{Oke:1983aa}.

\section{Data}\label{sec-data}

\begin{deluxetable}{lcccccccc}
\tabletypesize{\scriptsize}
\tablecaption{Description of the $z\sim1.6$ SpARCS cluster sample\label{tbl-clusters}}
\tablewidth{0pt}
\tablehead{
\colhead{Cluster}
& \colhead{R.A.}
& \colhead{Decl.}
& \colhead{z}
& \colhead{Spectroscopy}
& \colhead{Photometry}
& \colhead{Spectra$^\mathrm{a}$}
& \colhead{N$_\mathrm{spec}^\mathrm{b}$}
% & \colhead{N$_\mathrm{tot}^c$}
}
\startdata
SpARCS-J0224 & 02:24:26.33 & -03:23:30.8 & 1.633 & MOSFIRE, FORS2, OzDES & \textit{ugrizYJKs} [3.6] [4.5] [5.8] [8.0] & 187 & 52 \\
SpARCS-J0330 & 03:30:55.87 & -28:42:59.5 & 1.626 & MOSFIRE, FORS2, OzDES & \textit{ugrizYJKs} [3.6] [4.5] [5.8] [8.0] & 535 & 40 \\
SpARCS-J0225 & 02:25:45.55 & -03:55:17.1 & 1.598 & MOSFIRE, FORS2, OzDES & \textit{ugrizYKs} [3.6] [4.5] [5.8] [8.0] & 126 & 22 \\
SpARCS-J0335 & 03:35:03.58 & -29:28:55.6 & 1.369 & FORS2, OzDES & \textit{grizYKs} [3.6] [4.5] [5.8] [8.0] & 81 & 22 \\
\enddata
\tablenotetext{a}{Number of spectra.}
\tablenotetext{b}{Number of spectroscopically-confirmed cluster members.}
% \tablenotetext{c}{Total number of spectroscopic and photometric cluster members.}
\end{deluxetable}

The galaxy clusters studied in this work were identified using the Stellar Bump Sequence technique described in detail in \citet[][see also \citealt{Papovich:2008aa}]{Muzzin:2013aa}.
Four high-redshift cluster candidates (see Table \ref{tbl-clusters}) were identified within the \textit{Spitzer} Adaptation of the Red-Sequence Cluster Survey \citep[SpARCS;][]{Wilson:2009ws,Muzzin:2009jm} using a two-color cut on \textit{Spitzer} IRAC 3.6~$\mathrm{\mu}$m  - 4.5~$\mathrm{\mu}$m color and $z^\prime$ - 3.6 $\mathrm{\mu}$m color.
Spectroscopic follow-up was performed using the MOSFIRE \citep{McLean:2010aa,McLean:2012aa} spectrograph on the Keck Telescopes and the Focal Reduction and Imaging Spectrograph 2 (FORS2, \citealt{Appenzeller:1998aa}) on the European Southern Observatory (ESO) Very Large Telescope (VLT).
A small number of spectra were also obtained from the OzDES survey \citep{Yuan:2015aa,Childress:2017aa}.

Spectroscopic confirmation of these clusters was followed by collecting optical imaging data in $u^\prime\ g^\prime\ r^\prime\ i^\prime$ bands.
For SpARCS-J0330, SpARCS-J0224, and SpARCS-J0335, these data were taken with IMACS at Magellan/Baade, while for SpARCS-J0225 these data come from the Canada-France-Hawaii Telescope (CFHT) Legacy Survey (CFHTLS) which used MegaCam on CFHT.
All four clusters were imaged in near-infrared \textit{Y}-, \textit{J}-,  and \textit{Ks}-band with HAWK-I at VLT.
Our photometry also includes the IRAC data from the Spitzer Wide-area Extragalactic Survey \citep[SWIRE;][]{Lonsdale:2003ow} with additional deeper observations in IRAC 3.6 and 4.5 $\mathrm{\mu}$m bands as part of the \textit{Spitzer} Extragalactic Representative Volume Survey (SERVS), and $z^\prime$-band data from the SpARCS survey taken by the MOSAIC-II camera at the Cerro Tololo Inter-American Observatory (CTIO).
% For more details on the spectroscopic and photometric observations and reductions we refer the reader to \citet{Nantais:2016aa}.

\subsection{Photometric Catalogs}

As described in detail in \citet{Nantais:2016aa}, the imaging data were combined into a PSF-matched photometric catalog by first using Source Extractor \citep{sextractor} to detect sources in the \textit{K$_s$}-band data.
Astrometric and pixel-scale matching was performed on all images using SWarp \citep{swarp} prior to photometry.
PSF matching was performed using IRAF to generate convolution kernels before matching $u^\prime~g^\prime~r^\prime~i^\prime~z^\prime$~\textit{Y~J~Ks} band data to the poorest image quality among these bands.
Aperture photometry was performed using Source Extractor in dual-image mode and was corrected for Galactic extinction using \citet{1998ApJ...500..525S} dust maps and a \citet{2011ApJ...737..103S} extinction law.
Robust photometric errors were calculated by directly measuring the 1-$\sigma$ variation in background flux in randomly-placed apertures that do not contain any sources.

The resulting catalog has photometry in $u^\prime~g^\prime~r^\prime~i^\prime~z^\prime$~\textit{Y~J~Ks} and 3.6, 4.5, 5.8, 8.0 $\mu$m.
We perform an RA/DEC matching to the FORS2 and MOSFIRE spectroscopic data to associate spectroscopic redshifts to galaxies where possible.
Altogether there are 136 spectroscopically-confirmed members across the four clusters in this sample (see Table \ref{tbl-clusters}).

\subsection{Photometric Redshifts}\label{sec-eazy}

With the publicly-available photometric redshift code EAZY \citep{Brammer:2008uk}, we fit the broadband photometry of each object in our photometric catalog to a linear combination of seven basis templates derived from the prescription in \citet{Blanton:2007ft}.
These templates have been optimized for deep optical-NIR broad-band surveys, and this code was optimized specifically for \textit{K$_s$}-selected samples such as our own.
The output of this code includes the best-fit SED, a photometric redshift, and the photometric redshift probability distribution function of the object.
When a spectroscopic redshift is available, EAZY fixes the best-fit redshift to this value.

\subsubsection{Photometric Redshift Membership Criteria}\label{sec-members}

Due to instrument constraints, the sample of spectroscopically-confirmed cluster members is almost entirely composed of star-forming galaxies.
We therefore adopt the photometric cluster membership criteria used by \citet{van-der-Burg:2013aa} and \citet{Nantais:2016aa,Nantais:2017aa}, and consider galaxies to be cluster members if $(z_{\mathrm{phot}}-z_{\mathrm{cluster}})/(1+z_{\mathrm{cluster}}) \leq 0.05$.
This membership criteria attempts to avoid biasing our sample toward either star-forming or quiescent galaxies, while using a range in photometric redshifts that closely matches the scatter of our photometric redshifts $(\sigma \sim 0.04)$.
The choice of 0.05 does not drive the results of this work, and repeating the analysis for cutoff values between 0.05 and 0.1 does not change our conclusions.

This necessarily introduces some contamination by field galaxies due to uncertainty in the photometric redshift estimates.
The present work relies on ratios of the star-forming and quiescent population, which will be largely unaffected if this contamination is not biased toward either galaxy type.
A previous analysis by \citet{van-der-Burg:2013aa} of a comparable data set and method shows that the overall rate of false positives and negatives is small and largely insensitive to galaxy type at $z\sim1$, and field galaxy surveys find a field passive fraction of $\sim0.5$  \citep[][, see also Appendix \ref{sec-a-field}]{Muzzin:2013aa}, suggesting that this selection introduces minimal error to our conclusions.



% Galaxies with spectroscopic redshifts are considered to be cluster members if they have a velocity relative to the cluster of $\Delta$ v $\le 1500$ km s$^{-1}$.

\subsection{Rest-Frame Colors and UVJ Classification}\label{sec-UVJ}

% \mynote{williams $z=1$ cut}

We infer rest-frame absolute magnitudes for each cluster member by convolving its best-fit SED (derived using EAZY) with filter curves at the redshift of each galaxy.
We note that the span of the observed filters ensures that rest-frame magnitudes are interpolated from the available data, often overlapping with multiple observed passbands.

Using rest-frame \textit{U-V} and \textit{V-J} colors, we classify cluster members as \textit{UVJ}-star-forming or \textit{UVJ}-quiescent.
The classification is accomplished by dividing the space of rest-frame \textit{UVJ} colors into a star-forming and a quiescent region.
The cuts we use to define these regions have been empirically calibrated by \citet{Williams:2009tt}, being fine-tuned to maximally reflect the bimodality of galaxy populations out to $z\sim2.5$.
\textit{UVJ}-based classification is frequently used to classify star forming and quiescent galaxies in surveys when spectroscopic or morphological information is not available \citep{Wuyts:2007aa,Williams:2009tt,Whitaker:2011aa,Patel:2012ab,van-der-Burg:2013zn,Whitaker:2013rz,muzzin2013,Strazzullo:2013aa}.

In Figure \ref{fig-cmr}, we plot rest-frame \textit{U-V} vs \textit{M$_J$} color-magnitude diagrams for all cluster members in the sample, with inset rest-frame \textit{U-V} versus \textit{V-J} color-color diagrams.
Galaxies are colored according to their \textit{UVJ} classification, separating into a red-sequence and blue cloud.

\subsection{Stellar Masses and Dust Reddening}\label{sec-fast}

Using the publicly-available SED fitting code FAST \citep{Kriek:2009eq}, we fit the 12-passband photometry of each cluster to \citet[hereafter BC03]{Bruzual:2003by} stellar population synthesis templates.
FAST proceeds by generating a grid of synthetic SEDs of stellar populations at the redshift of each galaxy from the given population synthesis templates, for a range of star formation histories (SFH), ages, and masses, with possible additional variation in dust attenuation and/or metallicity.
Best-fit stellar populations are then selected from this grid by minimizing ${\chi}^2$ when comparing the synthetic SED to the observed broad-band photometry of a given galaxy.

For our grid of parameters, we use a range of ages from 100 Myr to 10 Gyr (excluding ages greater than the age of the universe at the observed redshift) and an A$_\mathrm{V}$ ranging from 0 to 3 mag with a Calzetti extinction law \citep{Calzetti:2001hh}.
Throughout, we assume an exponentially-declining star formation history, along with a Chabrier IMF \citep{chabrierimf} and fixed (solar) metallicity of 0.02.

In the \textit{U-V} versus \textit{M$_J$} color-magnitude diagram of Figure \ref{fig-cmr}, galaxies segregate into a blue cloud and red-sequence.
The colors of these two populations reflect the underlying bimodal distribution in star formation rate, but this picture is complicated by the presence of star-forming galaxies with dust-reddened colors.
We therefore find it illustrative to plot the dust-corrected \textit{U-V} versus \textit{M$_J$} color-magnitude diagram in Figure \ref{fig-cmr-dustsub}.
To correct the photometry for dust, we first calculate the dust extinction in \textit{U} and \textit{V} bands for each galaxy from the total \textit{V}-band extinction (A$_\mathrm{V}$, determined through SED fitting), using a Calzetti extinction law \citep{Calzetti:2001hh}.
We then subtract the contribution from dust from each galaxy's rest-frame \textit{U} and \textit{V} magnitudes to derive the dust-corrected values of these magnitudes and colors.
% \mynote{should i clarify my dust correction method here?}

Comparing Figures \ref{fig-cmr} and \ref{fig-cmr-dustsub}, we note that the red-sequence is mostly unaffected by dust subtraction, as the quiescent population generally exhibits little dust reddening to begin with.
The blue cloud becomes brighter, spanning a wider range in $M_J$, and exhibits decreased scatter in \textit{U-V} color.
The \textit{UVJ}-star-forming and \textit{UVJ}-quiescent populations separate more cleanly in color-magnitude space following dust subtraction, exposing the intermediate green valley.

\begin{figure*}
\centering \includegraphics[width=\textwidth]{figures/c2/revised/colormag.pdf}
\caption[Color-magnitude diagram for the high-redshift SpARCS cluster sample]{Rest-frame \textit{U}-\textit{V} versus absolute \textit{J} magnitude diagram for all photometric-redshift-selected cluster members of the four clusters in the sample (see Table \ref{tbl-clusters}).
The inset panels show rest-frame \textit{U-V} versus \textit{V-J} color-color diagrams, and galaxies are colored red (quiescent) or blue (star-forming) according to their \textit{U-V} and \textit{V-J} colors (see Section \ref{sec-UVJ}).
The mass completeness of our sample corresponds roughly to a magnitude limit of $M_J \lesssim -23$.
% The dashed line shows the location of a model $z=1.6$ color-magnitude relation with a formation redshift of $z_f=5$.
\label{fig-cmr}}
\includegraphics[width=\textwidth]{figures/c2/revised/colormag_dustsub.pdf}
\caption[Dust-corrected color-magnitude diagram for the high-redshift SpARCS cluster sample]{Dust-corrected rest-frame \textit{U-V} versus absolute \textit{J} magnitude diagram for the four clusters in our sample.
Galaxies are colored as in Figure \ref{fig-cmr}.
Photometry is corrected for dust using a Calzetti \citep{Calzetti:2001hh} extinction law with A$_\mathrm{V}$ determined from SED fitting (see Section \ref{sec-fast}).
Compared to Figure \ref{fig-cmr}, the blue cloud reaches brighter magnitudes and exhibits smaller scatter in \textit{U-V} color.
The separation between the \textit{UVJ}-star-forming and \textit{UVJ}-quiescent populations is more apparent following dust subtraction.
\label{fig-cmr-dustsub}}
\end{figure*}

\section{Analysis}\label{sec-analysis}

In this section we describe the method used to measure the quenching timescale $t_Q$.
In Section \ref{sec-model}, a toy model relates the number of star-forming, intermediate, and quiescent cluster members to a quenching timescale.
In Section \ref{sec-ellipses} we describe cluster member classification and counts.
In Section \ref{sec-r} we describe our clustercentric radial cut, and a background subtraction is described in Section \ref{sec-bkg}.
Section \ref{sec-error} describes how we derive confidence intervals for $t_Q$ with a Monte Carlo method.

\subsection{Environmental Quenching Model and Mass Completeness Limit}\label{sec-model}

A galaxy that is actively forming stars will have blue optical colors dominated by the bright contributions of short-lived O- and B-class stars.
After the onset of quenching, a galaxy's colors will become redder as these high-mass stars exhaust their hydrogen fuel and leave the main sequence, without new stars to replace them.
Eventually, a quiescent galaxy's color will reflect primarily the red colors of low-mass, long-lived main sequence stars and red giants.
We define the quenching timescale as the time since first infall after which galaxies are quenched.
In this section, we provide a conceptual summary of the method we use to measure $t_Q$, and refer the reader to Appendix \ref{sec-math} for details.

Recent work has shown that environmental quenching can be described by two principal timescales, a ``delay time" ($t_D$) and a ``fade time" ($t_F$) \citep{Wetzel:2013aa, McGee:2014aa, Mok:2014aa, Haines:2015aa, Balogh:2016aa, Fossati:2017aa}.
In our model, a star-forming (blue) galaxy that is accreted by the cluster will remain blue for a time $t_D$ following infall, after which the onset of quenching causes it to become an intermediate (green) galaxy.
The galaxy will remain green for a time $t_F$, until star formation has ceased and it is quiescent (red).
This model of environmental quenching is shown schematically in Figure \ref{fig-sfr}.
The total quenching time $t_Q$, defined as the length of time after accretion until a galaxy is completely quenched, is then $t_D + t_F$.

\begin{figure}[h!]
\centering \includegraphics[width=\textwidth]{figures/c2/sfr_model.pdf}
\caption[Toy model of star formation history for a quenched galaxy]{Model of galaxy star formation rate as a function of time since infall. In this model, galaxies are star-forming and blue before being accreted by a cluster. They remain blue for a time $t_D$, the delay time, before they start to quench and become green. After a further time $t_F$, the fade time, star formation has ceased and the galaxy becomes red.
\label{fig-sfr}}
\end{figure}

We assume that infalling galaxies are accreted from the field.
Not every galaxy accreted from the field will be star-forming, especially at higher stellar masses, and lower redshifts.
We wish to eliminate from consideration those galaxies that were quenched in the field before they were accreted by the cluster.
It is possible to account for this by removing a fraction of quiescent galaxies proportional to the field quiescent fraction.
This fraction can be calculated from the COSMOS/UltraVISTA field galaxy mass functions computed by \citet{Muzzin:2013ab}.

Following the above considerations, the observed number counts of blue and green galaxies in a cluster are proportional to the length of time a galaxy spends in the delay and fade phases.
For example, a long fade time would make it easier to catch galaxies in the process of quenching, leading to larger observed numbers of green galaxies in a cluster.
To fully quantify these timescales, one needs to control for the galaxy accretion rate of a cluster, as a higher accretion rate leads to larger numbers of all types of galaxies.
With the added assumption of a cluster galaxy infall rate, the number counts of red, green, and blue galaxies can constrain the timescales $t_D$ and $t_F$.

Given that blue galaxies have not resided in the cluster any longer than one $t_D$, their number will be equal to the cluster galaxy accretion rate $\mathrm{d}N/\mathrm{d}t$ integrated between the time of observation and one $t_D$ earlier.
In a similar manner, the number of green galaxies will be equal to the galaxy accretion rate integrated between one $t_D$ and one $t_D+t_F$ earlier.
The red galaxies trace all mass accreted earlier than one $t_D+t_F$ ago.
We write

\begin{empheq}{align*}
&B = \int_{-t_D}^{0} \mathrm{d}N/\mathrm{d}t\ \mathrm{d}t\\
&G = \int_{-(t_D+t_F)}^{-t_D} \mathrm{d}N/\mathrm{d}t\ \mathrm{d}t\\
&R = \int_{-t_H}^{-(t_D+t_F)} \mathrm{d}N /\mathrm{d}t\ \mathrm{d}t
\end{empheq}

where R,G, and B are the number of red, green, and blue galaxies respectively, $t_H$ is the Hubble time, and negative signs indicate that these galaxies were accreted in the past.

We assume that the cluster galaxy accretion rate $\mathrm{d}N/\mathrm{d}t$ is proportional to the cluster halo mass accretion rate $\mathrm{d}M/\mathrm{d}t$ as derived from the Millennium-\textsc{II} simulation by \citet{Fakhouri:2010aa}.
From there, ratios of the observed numbers of R, G, and B galaxies can be related to $\mathrm{d}M/\mathrm{d}t$, $t_F$, and $t_D$, to constrain the fade and delay times and thereby the total quenching time.
In Appendix \ref{sec-math} we more fully describe this toy model, which is ultimately described by a set of four equations, \eqref{eq-model-first}~--~\eqref{eq-model-last}.
Given a number of R, G, and B galaxies, a cluster redshift, and a mass accretion rate, Equations \eqref{eq-model-first}~--~\eqref{eq-model-last} can be solved for $t_F$, $t_D$, and $t_Q$.

Before proceeding with the analysis, we note several considerations which must be taken into account with this model.
The $80\%$ mass completeness of our sample is defined as the lowest mass for which passive galaxies, visible at the \textit{K$_s$} limit, yield accurate passive fractions \citep{van-der-Burg:2013zn}.
This limit varies from $10^{10.3}$ to $10^{10.5}\ \mathrm{M}_\odot$ within our sample \citep{van-der-Burg:2013zn,Nantais:2017aa}, due to variations in exposure time.
We must restrict our analysis to galaxies with masses above these limits, to ensure a fair comparison of the numbers of quenched and not-yet-quenched galaxies.

Second, it has been shown that the environmental quenching timescale varies with satellite galaxy mass \citep{De-Lucia:2012aa,Wetzel:2013aa,Wheeler:2014aa,Fillingham:2015aa}, and it is therefore inaccurate to refer to a singular environmental quenching timescale for all galaxies.
Any quenching timescale measured with the above toy model will necessarily be for an ensemble of galaxies spanning some range in stellar mass.
However, the quenching timescale does not vary much over the small dynamical range in mass studied in this work, at least at low redshift \citep[e.g. see Fig. 8 of ][Figure 5]{Fillingham:2015aa,Wetzel:2013aa}.

Third, the mass dependence must be considered when comparing with results of different studies.
Comparing with other studies will allow us to investigate the evolution of $t_Q$ with redshift (see Section \ref{sec-discussion}).
Other measurements of $t_Q$ will not be comparable to our results unless they were derived for a similar mass range.

For the above reasons, when measuring $t_Q$ we restrict our sample to galaxies with stellar masses above a mass completeness limit $M_* \ge 10^{10.5}~ \mathrm{M}_\odot$ ($M_J\sim-23$).
% \mynote{here's a place where i cite the data points in the final figure.}
This cut conservatively ensures that we are sampling above the mass completeness of our photometry for each cluster, and allows comparison with various results in the literature that report the quenching timescale for this range of masses.

In general, environmental quenching is likely the result of many different mechanisms operating over different timescales and environments.
A toy model such as the one presented here is not intended to be a final description of environmental quenching, but instead to investigate which physical scenarios, if any, are consistent with a set of very simple assumptions.

\subsection{Classification of Galaxies as Star-Forming, Intermediate, or Quiescent}\label{sec-ellipses}

The environmental quenching model described in Section \ref{sec-model} and Appendix \ref{sec-math} relates the number of observed star-forming (blue), intermediate (green), and quiescent (red) cluster members to the delay and fade times, $t_D$ and $t_F$.
A method of classifying galaxies as red, green, or blue is therefore needed before we can solve for the quenching timescale, $t_Q$.

A common approach to identifying star-forming, intermediate, and quiescent galaxies is to categorize them according to their colors and magnitudes, in a manner informed by galaxy evolutionary models.
A successful classification scheme will distinguish between star-forming galaxies that appear red due to dust, and galaxies that are red from a lack of star formation.
In this section we introduce a classification based on dust-corrected rest-frame colors derived from SED fitting (see Sections \ref{sec-UVJ} and \ref{sec-fast}).

Each galaxy's best-fit SED parameters include the \textit{V}-band dust reddening $\mathrm{A_V}$, which we use in conjunction with a Calzetti extinction law \citep{Calzetti:2001hh} to determine the reddening in \textit{U}- and \textit{B}-bands.
Subtracting this reddening from the rest-frame photometry breaks the color degeneracy between dusty, star-forming galaxies and old, quiescent galaxies.
Following dust-subtraction, galaxies separate more cleanly into a red-sequence, green valley, and blue cloud in a color-magnitude diagram, such as those shown in Figure \ref{fig-cmr-dustsub}.
We can therefore use cuts in dust-corrected color-magnitude space to label galaxies red, green, or blue.

To define these cuts, we start by applying a spectral clustering algorithm to the dust-corrected color-magnitude diagram of all galaxy cluster members.
This algorithm labels the two principal clusters of data points, identified in this case with the blue cloud and red-sequence.
We then fit an elliptical region to each cluster of data points by finding the eigenvectors of the covariance matrix of the set of points, which define the semi-major and semi-minor axes of an ellipse.
The width and height of this ellipse are scaled so that the ellipse represents a 95\% ($\operatorname{3-\sigma}$) confidence level.\footnote{Specifically, the length of each axis is $6\sqrt{\lambda}$, where $\lambda$ is the eigenvalue of the axis's eigenvector.}
Galaxies are categorized as either star-forming or quiescent according to their membership in these elliptical regions.
We define the green valley as the overlapping area of these ellipses, and galaxies within this region are categorized as intermediate.
In Figure \ref{fig-ellipses} we plot the classification regions over the dust-corrected rest-frame colors and magnitudes of all cluster members.

Included on this plot is a BC03 evolutionary track for a stellar population with a star formation rate that remains constant for 6 Gyr, after which it truncates (quenches).
There is a clear agreement between the model's stage of evolution and its progressive classification from blue, to green, to red.
In its star-forming phase, a galaxy stays in the blue region, and doesn't enter the green (intermediate) region until it is quenched.
After quenching, the model crosses the green valley in $\sim0.2$ Gyr.
The straightforward nature of galaxy evolution in this dust-subtracted color-magnitude space is the primary advantage of this classification scheme, which identifies an unambiguous green valley between the blue cloud and red-sequence.

% how to introduce the subject of final tQ? is this the right place to put it?
These elliptical regions define the star-forming, quiescent, and intermediate populations, and therefore the final value of $t_Q$ depends on their precise contours.
The total value of $t_Q$ is set by the location of the border between the green and red population, while the blue-green border, determining the fraction of star-forming galaxies that are intermediate, affects the way $t_Q$ is subdivided into $t_D$ and $t_F$.
Through repeated experimentation, we determine that the results are robust to reasonable tweaks in the contours of these ellipses, which affect the result within error bars.
The red-green border necessarily lies in the green valley, a region of low galaxy number density, where tweaks to this border do not affect the bulk of galaxies considered quenched or star-forming, therefore having little effect on their ratio.

\begin{figure}[h!]
\centering \includegraphics[width=\textwidth]{figures/c2/ellipses_clustering.pdf}
\caption[Criteria for classifying star-forming, intermediate, and quiescent galaxies based on cuts in dust-corrected color-magnitude space]{Classification of star-forming, intermediate, and quiescent galaxies.
We plot the dust-corrected rest-frame \textit{U-B} versus absolute \textit{B} magnitude for all cluster members.
Points are colored according to galaxies' \textit{UVJ} classifications (see Section \ref{sec-UVJ}).
The colored lines show $3\mhyphen\sigma$ elliptical fits to the two principal clusters of data points identified by a spectral clustering algorithm.
The elliptical regions define the quiescent, intermediate, and star-forming populations of galaxies, as labeled.
The solid black line is a BC03 model evolutionary track for continual star formation that truncates after 6 Gyr.
The black line is punctuated by dashes indicating time intervals evenly spaced in redshift.
The black points on this line mark when the model is is $0.10$ and $9.13$ Gyr old.
This track demonstrates good agreement between the model's star formation rate and its progressive classification from blue, to green, to red.
Note that even after 6 Gyr of constantly-integrated star formation, the model remains fully within the star-forming ellipse, only leaving it after quenching.
\label{fig-ellipses}}
\end{figure}

\subsubsection{UVJ-based Classification}

Rest-frame \textit{U-V} versus \textit{V-J} color-color diagrams can also be used to classify star-forming and quiescent galaxies (see Section \ref{sec-UVJ}), suggesting the possibility of using \textit{UVJ} classification in place of the dust-corrected color-magnitude classification of Section \ref{sec-ellipses}.
However, the location of the green valley in \textit{UVJ} space is not clear.

This alone is not fatal to the prospect of applying \textit{UVJ} classification to the quenching model, as the model can be simplified to forgo the use of green galaxies.
This simplification comes at the cost of being unable to constrain separate delay and fade times $t_D$ and $t_F$, instead directly measuring the total quenching time, $t_Q$.
When this simplified model is used in conjunction with number counts of \textit{UVJ}-star-forming and \textit{UVJ}-quiescent galaxies, the result agrees with that derived for the above dust-corrected color-magnitude classification.

For details on both of these points, we refer the reader to Appendix \ref{sec-a-uvj}.
This subject will be further elaborated in a letter (Foltz 2017, in prep).

\subsection{Projected Radial Distance Cut}\label{sec-r}

A cluster galaxy's type is known to correlate with clustercentric distance.
We wish to compare and combine galaxy number counts across multiple clusters and cluster samples, and therefore must control for galaxies' locations within the cluster.
Although a cut based on galaxies' positions relative to the cluster's virial radius is commonly used for this purpose, it is unlikely that the clusters in the high-redshift cluster sample are virialized structures.
Although we can measure a velocity dispersion for these clusters, it would not be meaningful to ascribe virial radii to them.

We therefore test our method using a variety of cuts on physical clustercentric distance, $r \leq 1000$ kpc, $r \leq 1500$ kpc, and $r \leq 2000$ kpc.
The choice of radial cut does not greatly affect the results of our analysis, and so we choose to restrict our consideration to galaxies with $r \leq 2000$ kpc.

\subsection{Background Subtraction}\label{sec-bkg}

Our number counts are contaminated by the inclusion of field galaxies due to the uncertainty inherent in photometric-redshift selection.
Before determining the quenching timescale we need to subtract the field galaxy background.
We therefore adjust the number counts for each cluster to correct for field contamination estimated from the field galaxy survey catalogs from UltraVISTA/COSMOS \citep{muzzin2013}.

To estimate the number of field galaxies included in the cluster sample, we start by cropping a randomly-selected section of the Ultra-VISTA/COSMOS dataset to match the angular size of the cluster photometry.
We process the Ultra-VISTA/COSMOS photometry with EAZY and FAST (see Sections \ref{sec-eazy} and \ref{sec-fast}) to determine photometric redshifts, rest-frame colors, and masses, limiting the data set to the same photometric bands that are available in the main dataset.
We then select field galaxies from this sample at the redshift of the cluster based on the same photometric redshift criterion described in Section \ref{sec-members}.
These field galaxies are classified as star-forming, intermediate, or quenched, according to the dust-corrected color-magnitude cuts detailed in Section \ref{sec-ellipses}.
We then subtract these numbers of red, green, and blue field galaxies from the corresponding numbers of cluster galaxies.

\subsection{Uncertainty Calculation}\label{sec-error}

Shown in detail in Appendix \ref{sec-math}, the numbers of red, green, and blue cluster galaxies, together with a cluster redshift, fully determine a quenching timescale.
The uncertainty in $t_Q$ is driven by uncertainty in these number counts, and we therefore use a Monte Carlo method to estimate the $68\%$ confidence interval for $t_Q$.

For each cluster, we create 200 simulated data sets consisting of numbers of red, green, and blue number counts, each drawn from Poisson distributions centered on the background-subtracted numbers of red, green, and blue galaxies.
For each simulated data set we substitute the values for R, G, and B into Equations~\eqref{eq-model-first}~--~\eqref{eq-model-last} and solve for $t_Q$, arriving at a distribution in $t_Q$.
The central $68\%$ of this distribution then defines the upper and lower confidence intervals for $t_Q$.

\section{Results}\label{sec-results}

\begin{deluxetable}{ccccccccc}
\tabletypesize{\scriptsize}
\tablecaption{Quenching timescale measured in the SpARCS and GCLASS cluster samples\label{tbl-results}}
\tablecolumns{9}
\tablewidth{0pt}
\tablehead{
\colhead{Cluster} \vspace{-0.4cm} & & & & & & \colhead{$t_D$} & \colhead{$t_F$} & \colhead{$t_Q$} \\ \vspace{-0.4cm}
& \colhead{N$^\mathrm{a}$} & \colhead{$\bar{z}$} & \colhead{R$^\mathrm{b}$} & \colhead{G$^\mathrm{b}$} & \colhead{B$^\mathrm{b}$} & & & \\
\colhead{Sample} & \colhead{} & \colhead{} & \colhead{} & \colhead{} & \colhead{} & \colhead{(Gyr)} & \colhead{(Gyr)} & \colhead{(Gyr)}
}
\startdata
GCLASS & 10 & 1.04 & 160 & 42 & 38 & \gtd & \gtf & \gresult\\
SpARCS high-redshift & 4 & 1.55 & 79 & 17 & 63 & \htd & \htf & \hiresult\\
\enddata
\tablenotetext{a}{Number of galaxy clusters in the sample.}
\tablenotetext{b}{Number of red, green, or blue galaxies above the mass completeness limit.}
\end{deluxetable}

Here we report the results of the quenching timescale modeling described in Section \ref{sec-model}.
In Section \ref{sec-z1.6} we report the measured quenching timescale for our high-redshift sample.
In Section \ref{sec-z1} we report the quenching timescale measured in a sample of galaxy clusters at $z\sim1$, and compare with a previous, independent measurement of the same reported by \citet{Muzzin:2014aa}.

The results are summarized in Table \ref{tbl-results}.

\subsection{Quenching timescale at $z=1.55$}\label{sec-z1.6}

We start by selecting cluster members according to the photometric redshift probability cut defined in Section \ref{sec-members}.
We classify galaxies as red, green, or blue according to their colors and magnitudes by the method described in Section \ref{sec-ellipses}.
We stack the sample by taking the total number of red, green, and blue galaxies at the mean redshift of the cluster sample, $z_c=1.55$.
We substitute these values for R, G, B, and $z_c$ into Equations~\eqref{eq-model-first}~--~\eqref{eq-model-last} and solve for $t_Q$, finding a quenching timescale of $t_Q=$\hiresult Gyr for this sample.
% Error bars are calculated as in Section \ref{sec-error}.

\subsection{Quenching timescale at $z=1.0$}\label{sec-z1}

The Gemini Cluster Astrophysics Spectroscopic Survey \citep[GCLASS,][]{Muzzin:2012dw} is a sample of 10 red-sequence-selected clusters at $0.87 < z < 1.34$, initially detected by the SpARCS optical/IR cluster survey using the cluster red-sequence detection method developed by \cite{Gladders:2000rq} \citep[see][]{Muzzin:2009jm,Wilson:2009ws,Demarco:2010om}.
GCLASS forms a complimentary data set to the $z\sim1.6$ SpARCS sample, having a similar range of optical to far-infrared photometry and catalogs prepared in a homogeneous manner \citep[see][]{Muzzin:2012dw,van-der-Burg:2013zn,Nantais:2016aa,Nantais:2017aa}.
With this data set, we can compare quenching timescales at $z\sim1.6$ and $z\sim1$.

Using the GCLASS spectroscopic and photometric catalogs, we performed the same cluster member selection and categorization of Sections \ref{sec-members} and \ref{sec-ellipses}.
We then use the total number of red, blue, and green galaxies above the mass completeness limit $M_* \geq 10^{10.5}~\mathrm{M}_\odot$ to measure a quenching timescale according to Equations~\eqref{eq-model-first}~--~\eqref{eq-model-last}, finding $t_Q=$\gresult Gyr at $z\sim1$.

A previous analysis by our team has independently measured the quenching timescale in this sample.
\citet{Muzzin:2014aa} identified spectroscopic cluster members with absorption line features indicative of recent, rapidly-truncated star formation.
The distribution of these ``post-starburst" galaxies in phase space, when compared with the phase space of zoom simulations, indicated a quenching timescale of $\sim1\ \rpm\ 0.25$ Gyr.
This result is largely independent of the measurement performed in this present work, as it was derived using galaxies' spectroscopic features and positions within the cluster.
The agreement between these methods is therefore a strong indicator that they independently measure the same timescale, corresponding to the quenching time.

\section{Discussion}\label{sec-discussion}

\begin{figure*}[h!]
\centering \includegraphics[width=\textwidth]{figures/c2/quenching_timescales.pdf}
\caption[Quenching timescale as a function of redshift]{
Quenching timescale as a function of redshift.
Red points show the quenching timescales measured for our cluster samples at $z\sim1$ and $\sim1.6$ (see Section \ref{sec-results}).
Black points show the quenching timescales measured in clusters by \citet{Wetzel:2013aa}, \cite{Muzzin:2014aa}, \citet{Taranu:2014aa}, \citet{Haines:2015aa}, and \citet{Balogh:2016aa}.
Hollow gray points indicate quenching timescales measured in groups by \citet{McGee:2011aa}, \citet{Balogh:2016aa}, and \citet{Fossati:2017aa}.
All data points are for galaxies with $M_* \ge 10^{10.5}~ \mathrm{M}_\odot$.
The dashed green line represents the evolution of a dynamical timescale normalized to 7 Gyr at $z=0.05$, the quenching time in SDSS groups as reported by \citet{Balogh:2016aa}.
The shaded green region represents the evolution of the dynamical timescale normalized to $5.0\pm0.5$ Gyr, spanning the range of quenching times in SDSS clusters as reported by \citet{Wetzel:2013aa} and \citet{Balogh:2016aa}.
The solid red line indicates an approximation of the total gas depletion timescale, $t_{\mathrm{depl}}(\mathrm{H_I}+\mathrm{H_{mol}})$, adapted from the molecular gas depletion timescale measured by \citet[][see text]{Tacconi:2017aa}.
The solid blue and orange lines are estimates of the quenching time in an SFR outflow scenario.
The blue line is an estimate of the gas depletion timescale with a mass loading factor of $\eta=2.5$, described by \citet{McGee:2014aa}.
The orange line shows the global evolution in star formation rates of the fundamental plane as measured by \citet{Whitaker:2012aa}, which approximates the evolution of an outflow gas depletion time, normalized to the low-redshift time of 5 Gyr.
\label{fig-tq}}
\end{figure*}

Based on the results of Section \ref{sec-results}, the quenching timescale for massive satellite galaxies ($M_* \ge 10^{10.5}~ \mathrm{M}_\odot$), measured in a homogeneous manner across cluster samples, is $\simgresult$ Gyr at $z\sim1$ and $\simhiresult$ Gyr at $z\sim1.6$.
These quenching times are required to produce the observed number of quenched galaxies in our cluster sample, given a reasonable model of the mass accretion histories of clusters.
We plot the evolution of the cluster quenching timescale with redshift in Figure \ref{fig-tq}.

\subsection{Redshift Evolution of Observed Quenching Timescales}

Included on Figure \ref{fig-tq} are several quenching timescales drawn from other studies.
We note several possible sources of confusion that must be accounted for when drawing fair comparisons between timescales reported in the literature.
Historically, researchers have used several different approaches to modeling or measuring the quenching timescale, and occasionally even different definitions of the quenching timescale itself.
We have taken $t_Q$ to be the time following infall for a galaxy to be classified quiescent, and following \citet{Wetzel:2013aa}, describe it by a ``delayed" followed by a ``fade" phase.
Other formalisms have been adopted, such as ``slow quenching" scenarios where galaxies begin quenching immediately upon infall, having star-formation rates that decline gradually with an exponential time constant (often also called the ``quenching time").

These considerations are additional to the normal systematic differences in galaxy samples and completenesses, classification systems, membership selections, and background subtractions.
In the end, all models must necessarily employ various simplifying assumptions, and are approximations to a full description of galaxy quenching.
% Nevertheless, meaningful comparisons can be drawn between the various measurements of $t_Q$, and information can be extracted from the redshift evolution of this timescale.

The data points described here were all measured for group or cluster galaxies in stellar mass ranges equal or comparable to our mass completeness limit, $M_*~\ge 10^{10.5}~ \mathrm{M}_\odot$.
In works where the quenching timescale was reported for separate redshift or mass bins, we take the mean quenching timescale for galaxies above our mass limit, at the mean redshift of the redshift bin.
We plot cluster measurements as solid black symbols, while group measurements are plotted as hollow gray symbols.

\citet{McGee:2011aa,McGee:2014aa} studied the passive fraction in galaxy groups taken from the Group Environment Evolution Collaboration \citep[GEEC and GEEC2,][]{Balogh:2014aa}.
\citet{McGee:2014aa} relates the group passive fraction of $\sim~0.3$ at $z=0.4$ to infall histories in semi-analytic simulations \citep{McGee:2009aa}, where 30\% of galaxies became satellites more than $4.4\pm0.6$ Gyr ago.
From this, it is concluded that the quenching time for these galaxies is 4.4 Gyr.

This basic approach was adapted by \citet{Wetzel:2013aa}, \citet{Balogh:2016aa}, and \cite{Fossati:2017aa}, and applied to galaxy groups and clusters in the Sloan Digital Sky Survey \citep[SDSS,][]{York:2000aa}, GEEC2, GCLASS, and deep-field 3D-HST/CANDELS \citep{Grogin:2011aa,Koekemoer:2011aa,Brammer:2012aa} data sets.
In SDSS clusters, \citet{Wetzel:2013aa} find a total quenching time of $4.4 \pm 0.4$ Gyr, where \citet{Balogh:2016aa} finds $5.0\pm0.5$ Gyr.
\citet{Balogh:2016aa} also finds a quenching time of $7.0 \pm 0.5$~Gyr in SDSS groups, $2.8 \pm 0.5$~Gyr in GEEC2 groups, and $1.5 \pm 0.5$~Gyr in the GCLASS cluster sample.
\citet{Fossati:2017aa} reports the quenching timescale for groups in the 3D-HST/CANDELS fields in three redshift bins spanning $0.5 < z < 1.80$, finding quenching times between 2 and 3 Gyr.

\citet{Muzzin:2014aa} employ a different method to constrain quenching timescales in the GCLASS cluster sample.
Using galaxy spectral features, they identify a population of poststarburst galaxies.
The distribution of this population in cluster phase space\footnote{``Cluster phase space" here refers to the phase space spanned by galaxies' velocities relative to the cluster and their projected clustercentric radius.} can be related to the evolving phase space distribution of infalling subhalos in dark-matter zoom simulations to determine a timescale.
\citet{Muzzin:2014aa} reports that this process indicates a rapid fade time of $t_F \simeq 0.5$ Gyr following the galaxy's first pass through $0.25\mhyphen0.5~ R_{200}$, a passage which requires a time $t_D=0.45\pm0.15$~ Gyr in the simulations, for a total quenching time of $t_Q=1.00\pm0.25$~ Gyr.

Other studies have successfully measured quenching timescales, but use different models or assumptions that complicate direct comparison with the present work.
While we define $t_Q$ to be the time after accretion required for a galaxy to be classified quiescent, it is not uncommon to find the quenching timescale defined in other ways.
In the ``slow quenching" model, star-formation rates decline gradually with an exponential time constant $\tau_Q$ starting immediately upon infall.
To convert from this framework to our present system of classification, we create BC03 model stellar populations with star formation rates that remain constant until infall, after which they decline with time constant $\tau_Q$.
We then plot the evolution of the model rest-frame colors and magnitudes on the classification ellipses of Figure \ref{fig-ellipses}, and take $t_Q$ to be the time required after infall before the model is considered red.

\citet{Haines:2015aa} employ a similar phase-space approach to \citet{Muzzin:2014aa}, comparing the radial density profiles of star-forming galaxies in clusters at $z\sim0.2$ to the evolving radial densities of infalling halos in clusters the Millennium-\textsc{II} simulation, at a slightly lower mass completeness limit of $2\times10^{10}~ \mathrm{M}_\odot$.
They adopt the ``slow quenching" model, and find the kinematic properties of the star-forming population to be best fit by an exponentially-declining star formation rate with time constant $\tau_Q=1.73\pm0.25$~ Gyr.
The value of $t_Q$ corresponding to this result depends on the assumed age of the galaxy at time of infall.
Cluster red-sequence galaxies at $z \lesssim 1$ have colors consistent with having been formed at $z \gtrsim 3$ \citep{Foltz:2015aa}, and models of cluster mass-accretion rates suggest that a typical halo in a cluster at $z=0.2$ was accreted at $z\sim1.1$ \citep{Fossati:2017aa}.
Therefore we construct our model with an age of 3 Gyr at infall, and find that $\tau_Q=1.73\pm0.25$~Gyr corresponds to $t_Q\simeq3.7\pm0.5$~Gyr.

\citet{Taranu:2014aa} employ a novel combination of observed galaxy bulge and disc colors, models of quenching star formation rates, and subhalo orbits drawn from cosmological N-body simulations.
They too adopt a ``slow quenching" model, and their data are best fit by an exponentially-declining star formation rate time constant of $\tau_Q=3\mhyphen3.5$~Gyr, with quenching beginning immediately upon infall.
Adopting the same conversion method as we use for \citet{Haines:2015aa}, we find this corresponds to $t_Q\simeq4\pm2$ Gyr.
We note that \citet{Taranu:2014aa} use a sample of brightest cluster (and group) galaxies, an extremal population of quenched galaxies, for which our model likely breaks down.

Other notable studies preclude comparison with the present work, due to differences in mass completeness, or differences in analysis.
\citet{Oman:2016aa} use a phase space approach to characterize the quenching timescale in SDSS clusters.
\citet{Oman:2016aa} derive orbital histories for cluster and satellite galaxies from dark-matter simulations, characterizing the probability that each galaxy becomes quiescent as a function of time, $p_q(t)$.
They report a typical delay time of $t_D=3.5\mhyphen5$~Gyr and a $p_q(t)$ that evolves with a time constant $\tau\lesssim2$~Gyr.
We do not attempt to interpret this in terms of a $t_Q$ value.

\citet{Gobat:2015aa}, studying galaxies of mass $M_* \gtrsim 10^{11}~ \mathrm{M}_\odot$ in groups in the COSMOS field at $z\sim1.8$, find evidence for a rapid fade time of $t_F\approx0.3$~Gyr, based on the properties of satellite galaxies.
\citet{Paccagnella:2016aa,Paccagnella:2017aa} study poststarburst galaxies in cluster phase space at low redshift and conclude that quenching happens by separate rapid and slow-quenching scenarios, traced by poststarbursts and intermediate galaxies, respectively, at a lower mass completeness of $M_* \gtrsim 10^{9.8}~\mathrm{M}_\odot$.
They find that intermediate galaxies are described by a slow-quenching scenario with a total timescale of 2-5~Gyr, although fast quenching produces two times as many passive galaxies.

\subsection{Remarks on Methods and Systematic Error}

The various techniques that have been used all share two main features in common.
First, they all must label a population of quenched galaxies, and/or a star-forming population.
This is accomplished variously by cuts on colors and/or magnitude, inferred star formation rates, or galaxy spectral features.
Second, they must relate the characteristics of the quenched or active population, or quenched fraction to timescale information.
This is universally done by comparison with numerical simulations, which can relate infall times to distributions in phase space, radial surface densities, or to mass accretion histories, as in the present work.

Besides these fundamental differences in model, the next most important source of systematic error is likely the choice of how to treat the field-quenched correction (Appendix \ref{sec-a-field} in the present work).
When characterizing the quenched population of a cluster, one needs to account for the fact that the observed quenched fraction in clusters isn't entirely the result of quenching within the cluster, because quenched galaxies are found in the field as well.
Therefore some number of quenched galaxies need to be subtracted from the observed count, in a manner informed by the field quenched fraction.
For \citet{McGee:2011aa}, \citet{Balogh:2016aa}, and \cite{Fossati:2017aa}, this is done by calculating the quenched fraction that is in excess of the field at the observed redshift of the cluster, which is sometimes referred to as the ``conversion fraction" or the ``environmental quenching efficiency".
The approach used by \citet{Wetzel:2013aa} and the present work is to instead subtract off those field galaxies that were quenched at the time of accretion, not at the time of observation.

As explained in Appendix B of \citet{Balogh:2016aa}, the different approaches amount to a philosophical difference about what is being measured.
By calculating the conversion fraction, one removes not only those galaxies which were quenched at the time of accretion, but also those which would have quenched in the field by the time of observation, too.
The result is that the \citet{Wetzel:2013aa} approach measures the time taken for galaxies to quench in dense environments, while the ``conversion fraction" approach measures the timescale due purely to environmental quenching.
\citet{Balogh:2016aa} found $t_Q$ to be higher by 0.5~Gyr for SDSS clusters than previous estimates by \citet{Wetzel:2013aa}, and attributes this difference to the above difference in field subtraction methods, while noting that the true answer likely lies somewhere in between.
By $z\sim1$, $t_Q$ as measured in the GCLASS cluster sample by \citet{Balogh:2016aa} and the present work agree within error bars.

For the present work, the field correction approach of \citet{Wetzel:2013aa} is necessary.
Our model requires a direct comparison between quenched galaxies and those which have not yet been quenched, under the assumption that these populations are the same except for the time they have spent in the cluster.
In other words, the model assumes that the B, G, and R populations represent an evolutionary sequence, B $\rightarrow$ G $\rightarrow$ R.
It is possible to calculate the conversion fraction of our cluster sample (see \citealt{Nantais:2016aa,Nantais:2017aa}), arriving at the number of cluster galaxies quenched due solely to environment, but these would have to be compared to only those blue galaxies that will quench due solely to environment.
It is unclear how to correct the blue population in this way without knowing the quenching timescale in advance.
We therefore adopt the convention of subtracting only those galaxies that were already quenched at the time of accretion, and therefore measure the net change in galaxy properties since infall.

Of special interest within the assembled data set is a comparison between the three studies that have measured the quenching timescale in the GCLASS sample \citep[][and the present work]{Muzzin:2014aa,Balogh:2016aa}.
Specifically, at $z=1.05$, \citet{Muzzin:2014aa} finds $t_Q=1.00\pm0.25$ Gyr, the present work finds \gresult~Gyr, and \citet{Balogh:2016aa} finds $t_Q=1.5 \pm 0.5$ Gyr.
The results of \citet{Balogh:2016aa} are consistent within error bars with the present work, and \citet{Muzzin:2014aa} very nearly so.
Differences can be attributed to different approaches to measuring $t_Q$, including the above mentioned field corrections.
The definition of quenched galaxies differs as well, where \citet{Muzzin:2014aa} studies quenched poststarburst galaxies identified by their spectral features, \citet{Balogh:2016aa} uses an optical-IR color-color cut, and the present work uses a dust-corrected color-magnitude criteria.
Nevertheless, these three data points point clearly to a quenching time between 1 and 1.5~Gyr.
% It is possible that poststarburst galaxies are not representative of the entire population of quenched galaxies.

% While most studies discussed here used the properties of the entire population, paccaganella and muzzin target poststarburst galaxies specifically, and find tQ to be slightly lower in both cases.

% Since we base our results upon the population of galaxies accreted over time, tQ is time-averaged.
% The poststarburst phase is necessarily short-lived, and so if your method probes an integrated tQ, perhaps measuring these galaxies finds something closer to the instantaneous tQ.

\subsection{Redshift Evolution of Characteristic Timescales}

A clear evolutionary trend emerges from the assembled data points of Figure \ref{fig-tq}.
The quenching timescale at low redshift is long, roughly $4\mhyphen5$~Gyr, but has decreased to the order of $\sim1\mhyphen2$~Gyr at $z\sim1.5$.

Galaxy quenching may be the result of factors internal or external to the galaxy.
The former case includes scenarios where quenching occurs as a galaxy exhausts its gas reservoir (as in starvation, or overconsumption).
The latter case describes scenarios where quenching is due to the interaction of a galaxy with the host halo's environment at the high speeds typical of orbits within clusters.
In this section, we will endeavor to model several timescales associated with either gas depletion or kinematic effects, and plot them on Figure \ref{fig-tq}.

In gas depletion scenarios, the environment simply prevents cosmological accretion of fresh gas onto the galaxy, and what gas reservoir remains after infall is consumed by the galaxy over a gas depletion timescale $t_{\mathrm{depl}} = M_{gas}/\dot{M_{gas}}$, after which star formation ceases.
\citet{Fillingham:2015aa} note that measured molecular gas depletion timescales $t_{\mathrm{depl}}(\mathrm{H_{mol}})$ are much shorter than measured values of $t_Q$, over a broad range of redshifts.
This trend continues to be seen with the quenching timescales measured since the time of that study, including those in the present work.
In the local universe, however, \citet{Fillingham:2015aa} find very good agreement between the total gas depletion timescale $t_{\mathrm{depl}}(\mathrm{H_I}+\mathrm{H_{mol}})$ and the quenching times of high-mass galaxies ($M_* \geq 10^{9}~ \mathrm{M}_\odot$).
The first hypothesis we will consider is that the quenching timescale is simply the gas depletion timescale, where the galaxy's star-forming gas reservoir includes the atomic gas component.

A star-forming galaxy's molecular gas fraction is found to decrease slowly with redshift out to $z=2$, by roughly a factor of 2 \citep{Genzel:2015aa,Tacconi:2017aa}, while the atomic gas density remains nearly constant \citep{Bauermeister:2010aa}.
Since in the local universe, $M_{HI}\sim3M_{mol}$ \citep[see, e.g.,][]{Saintonge:2011aa}, for simplicty we will take $t_{\mathrm{depl}}(\mathrm{H_I}+\mathrm{H_{mol}})\sim~4~ t_{\mathrm{depl}}(\mathrm{H_{mol}})$, with the redshift evolution of $t_{\mathrm{depl}}(\mathrm{H_{mol}})$ from \citet{Tacconi:2017aa}, and plot it on Figure \ref{fig-tq}.

If galaxies experience significant star-formation-driven outflows, then the gas depletion timescale will be much shorter.
\citet{McGee:2014aa} has constructed a model parametrized by the ``mass-loading factor" $\eta$, such that the rate of gas mass ejected by a galaxy is a factor $\eta$ of the star formation rate.
We include on Figure \ref{fig-tq} the gas depletion time with outflows of $\eta=2.5$, using the cosmic evolution of the star formation rate derived by \citet{Whitaker:2012aa}.
This value of $\eta$ was found to best fit the quenching timescales described by \citet{McGee:2014aa}, and produces timescales that match $t_Q$ in clusters at low redshift.
While \citet{McGee:2014aa} intend for this timescale to model the delay time rather than the full quenching time, we include it on Figure \ref{fig-tq}, noting that any difference due to the addition of fade time can be recovered by minor adjustments to $\eta$.
It is broadly the case that outflow timescales for various values of $\eta$ scale with redshift approximately as SFR, and so we also include on Figure \ref{fig-tq} the SFR evolution of \citet{Whitaker:2012aa}, normalized to a low-redshift timescale of 5 Gyr.

The dynamical time $t_{\mathrm{dyn}}$ is commonly used to characterize timescales that depend on the kinematics of a galaxy within a cluster, such as gas stripping scenarios.
A cluster halo in virial equilibrium is characterized by relations between its radius $R$ and the velocity $V$ of its constituent galaxies, defining a dynamical timescale, $t_{\mathrm{dyn}}=R/V$.
From considerations of cosmology, the dynamical time is expected to scale with redshift as $t_{\mathrm{dyn}} \propto (1+z)^{-1.5}$.
If quenching is accomplished after a galaxy makes one or multiple passes through a particular radius of its host halo, $t_Q$ will be proportional to $t_{\mathrm{dyn}}$.
We normalize the dynamical timescale at low redshift separately to the SDSS group and cluster $t_Q$ data points.
We choose a normalization of $5.0\pm0.5$ Gyr for the cluster dynamical time scale, to span the two values for this data set reported by \citet{Wetzel:2013aa} and \citet{Balogh:2016aa}.
We normalize the group dynamical time scale to the 7 Gyr $t_Q$ reported by \citet{Balogh:2016aa}.
We plot these dynamical timescales also on Figure \ref{fig-tq}.

These trend lines roughly depict the expected evolution of $t_Q$ for various possible quenching scenarios.
They assume that the dominant quenching mechanism remains unchanged from low redshift, and is invariant for a given star formation rate and stellar mass.
We don't intend for these timescales to conclusively identify the mechanism responsible for environmental quenching, but rather to test if the measured redshift evolution of $t_Q$ is consistent with these possible models.

\subsection{Interpreting the Quenching Timescale}

The quenching timescale of massive galaxies ($M_* \geq 10^{10.5}~ \mathrm{M}_\odot$) is systematically higher in groups than in clusters.
In the SDSS sample, this trend is particularly pronounced, with $t_Q$ being higher in groups by $\sim2$ Gyr \citep{Balogh:2016aa}, although a difference is seen at all measured redshifts.
This difference cannot be entirely attributed to differences in background subtraction, as demonstrated by the agreement between the present work and \citet{Balogh:2016aa} for the GCLASS cluster sample.
If $t_Q$ truly exhibits a dependence upon the mass of the host halo, then the quenching timescale is driven in part by factors external to the galaxy.

The total gas depletion timescale $t_{\mathrm{depl}}(\mathrm{H_I}~+~\mathrm{H_{mol}})$ is a good fit for galaxies at low redshift, but seemingly evolves too slowly to be a good fit at higher redshift.
On the other hand, both estimates of an SFR-outflow timescale evolve too quickly at high redshift.
However, the cluster data points and group data points both evolve in accordance with the correspondingly-normalized dynamical timescale.
It is also possible that gas is simply consumed more quickly by high-mass galaxies at high redshift, as in an overconsumption scenario, although models with fixed mass-loading factor $\eta$ cannot simultaneously fit both the high- and low-redshift data points.

The evolution of the dynamical time represents an evolution in the properties of groups and clusters (velocity dispersions, halo masses, etc.), not galaxy properties (SFR, gas fractions, etc.).
If quenching tracks $t_\mathrm{dyn}$, then it must be determined by the dynamical properties of clusters .
Such a scenario is often interpreted as being evidential of dynamical quenching scenarios such as ram-pressure stripping.

The SFR-outflow timescale does not fit the data points at high redshift.
While $\eta$ may be adjusted to yield systematically longer or shorter timescales, it is the evolution of the timescale that makes it a poor fit to the data points, and this evolution is indeed driven by the evolution in galaxies' star formation rates.
We have used the SFR parametrization of \citet{Whitaker:2012aa} although other parameterizations exist \citep{Peng:2010aa}.
Fits to the cosmic evolution of the SFR can also vary depending on the assumed functional form \citep{Behroozi:2013aa}.
It is possible that a better fit to the measured values of $t_Q$ could be obtained through adjustments to the SFR-outflow model parameters used by \citet{McGee:2014aa}, although we do not attempt such an analysis here.

\citet{Balogh:2016aa} find that SFR-outflow quenching is a good fit to the delay times measured in the GCLASS and GEEC2 samples at $z\sim1$.
This conclusion is based primarily on the quenching timescales measured in galaxies with masses $M_* \leq 10^{10.3}~ \mathrm{M}_\odot$, which we do not study here.
For those galaxies, $t_Q$ is found to be longer by several Gyr, and to increase with decreasing galaxy mass, in a way that is well-modeled by SFR outflows with $1.0 \leq \eta \leq 2.0$, although the same model is a poor fit at low redshift.
\citet{Balogh:2016aa} report that the dynamical timescale is a good fit to $t_Q$ in galaxies with $M_* \geq 10^{10.5}~ \mathrm{M}_\odot$, as also noted by others \citep{Tinker:2010aa,Mok:2014aa}.
No disagreement is found between the present work and \citet{Balogh:2016aa} for the samples and analyses where these studies overlap.

In the local universe, \citet{Fillingham:2015aa} find that $t_Q$ in galaxies with masses $M_*~\geq~10^{10.5}~\mathrm{M}_\odot$ is well-fit by the total gas depletion timescale, $t_{\mathrm{depl}}(\mathrm{H_I}~+~\mathrm{H_{mol}})$.
Using a different estimate of $t_{\mathrm{depl}}(\mathrm{H_I}~+~\mathrm{H_{mol}})$ based on $t_{\mathrm{depl}}(\mathrm{H_{mol}})$ \citep{Tacconi:2017aa}, we arrive at the same conclusion.
However, the gas depletion timescale does not evolve quickly enough to reach times on the order of 1-2 Gyr at $z\sim1.5$.
The timescale we compare with here is a very rough extrapolation, as there are very few constraints on galaxy atomic gas budgets at $z\gtrsim0.1$ \citep[see discussion in ][]{Bauermeister:2010aa}.
Future work may better characterize the evolution of the total gas depletion timescale.

\section{Conclusions}\label{sec-conclusion}

In this paper we measured numbers of star-forming, intermediate, and quenched cluster members in two samples of galaxy clusters at $0.85 < z < 1.35$ and $1.35 < z < 1.65$.
A model of environmental quenching allows these number counts to constrain the quenching timescale $t_Q$.
From the analysis presented in this work, we draw the following conclusions:

\begin{itemize}

% \item
% We present a toy model and method for using number counts of galaxies in different stages of evolution to measure quenching timescales in clusters.

\item
We measure a quenching timescale of $t_Q= $\gresult{} Gyr in a sample of 10 galaxy clusters at $0.85 < z < 1.35$, while in a sample of 4 galaxy clusters at $1.35 < z < 1.65$ we find the quenching timescale to be $t_Q=$\hiresult Gyr.

\item
The evolution of the quenching timescale in clusters from the local universe to $z=1.55$ evolves faster than the gas depletion timescale but slower than an SFR outflow model.
Instead, it appears to scale with the dynamical time, when normalized to the quenching timescale in local galaxy clusters.
This suggests that kinematical quenching mechanisms such as ram-pressure stripping may dominate in galaxies with masses $M_* \geq 10^{10.5}~ \mathrm{M}_\odot$ in clusters at high redshift.


\item
The quenching timescale for galaxies with masses $M_* \geq 10^{10.5}~ \mathrm{M}_\odot$, measured out to $z\sim1.55$, appears to be shorter in clusters than in groups.
This indicates that environmental quenching mechanisms for these galaxies may depend on host halo mass at high redshift, as would be the case for kinematical quenching mechanisms such as ram-pressure stripping.

\end{itemize}

