\begin{abstract}
The accepted model of cosmology has been incredibly successful at matching observations of the large-scale structure.
The implications of this model are less understood on smaller galactic scales where astrophysical processes become important.
We seek to understand the processes of galaxy formation and evolution.
We present two studies of quiescent galaxies in cluster samples at 1 and 1.6.
We use the properties of the CMR to constrain formation periods for the red-sequence.
We use a simple quenching model to estimate a quenching timescale in both clusters.
Comparing the evolution of the quenching time with various models allows us to constrain various quenching scenarios.
\end{abstract}
