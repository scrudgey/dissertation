\begin{abstract}
We present two studies of galaxy populations in cluster samples at $z\sim1$ and $z\sim1.5$ with the aim of constraining galaxy formation and evolutionary processes.
We study the slope, intercept, and scatter of the color-magnitude relation in the ten clusters at z $\sim 1$.
The quiescent galaxies in these clusters formed the bulk of their stars above $z \gtrsim 3$ with an age spread $\Delta t \gtrsim 1$ Gyr.
We measure the environmental quenching timescale $t_Q$ in a sample of four galaxy clusters at $1.35 < z < 1.65$, and the $z\sim1$ sample.
We employ a ``delayed-then-rapid" quenching model that relates a simulated cluster mass accretion rate to the observed numbers of each type of galaxy in the cluster to constrain $t_Q$.
We find a quenching timescale of $t_Q=$ \hiresult Gyr in the $z\sim1.5$ cluster sample, and $t_Q=$ \gresult Gyr at $z\sim1$.
Using values drawn from the literature, we compare the redshift evolution of $t_Q$ to timescales predicted for different physical quenching mechanisms.
For galaxies of mass $M_* \gtrsim 10^{10.5}~ \mathrm{M}_\odot$, the environmental quenching timescale evolves faster from $z=0$ to $z=1.5$ than the gas depletion timescale and slower than an SFR-outflow timescale with mass-loading factor $\eta=2.5$, but is consistent with the evolution of the dynamical time.
This suggests that environmental quenching in these galaxies is driven by the motion of satellites relative to the cluster environment.
We also find $t_Q$ to depend on host halo mass such that quenching occurs over faster timescales in clusters relative to groups, further supporting the notion that kinematic mechanisms are responsible for quenching high-mass galaxies.
\end{abstract}
