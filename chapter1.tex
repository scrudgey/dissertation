\chapter{Introduction}

In recent years, the $\Lambda$CDM paradigm has emerged as the standard cosmological description of the universe.
In this model, the dynamical aspects of structure formation are well-described by the purely gravitational collapse of dissipationless dark matter, which can be studied in large N-body simulations \citep[e.g.,][]{Springel:2005aa,Thomas:1992aa,Navarro:1995aa,Weinberg:2004aa}.
Dark matter (DM) haloes then evolve through hierarchical assembly, being built-up from successive mergers of smaller haloes over cosmological times. \citep{Gott:1975ab,Press:1974aa,White:1978aa,Blumenthal:1984aa,Davis:1985aa,White:1991aa,Barnes:1992aa,Cole:2000aa}.
Baryonic matter traces the DM distribution, forming the visible components of these structures.
%Berlind & Weinberg 2002, Leauthaud et al. 2012b; Behroozi et al. 2013

The $\Lambda$CDM framework has enjoyed dramatic successes in accounting for the observed state of the universe, from the abundance and spatial distribution of the large-scale structures of the universe (clusters, voids, and filaments), to their emergence from the extremely smooth initial conditions indicated by the cosmic microwave background (CMB) \citep{Springel:2006aa}, as well as the temperature anisotropy of the CMB \citep[][and references therein]{Narlikar:2001aa}, and the abundances of the light elements.

On smaller, galactic scales (100 kpc), however, the consequences of this model are complicated by the dissipative physics of baryons:
radiative cooling of gas, star formation, active galactic nucleus (AGN) feedback, and chemical enrichment and energy release from supernovae (SN) are poorly-understood processes, which are currently parameterized by ad hoc recipes in galaxy formation models.
Consequently, the current generation of models have lacked a complete account of the build-up of stellar mass over cosmic time, and have difficulty matching the abundance of quenched galaxies \citep[e.g.][]{Weinmann:2011aa,Vulcani:2014aa} or the star formation rates of galaxies \citep[e.g.][]{Font:2008ab,Weinmann:2010aa}.
Better observational constraints on the formation and evolution of galaxies are vitally important in order to complete the description of the relevant physics, from large-scale cosmology to small-scale astrophysics.

% \mynote{figure of millennium structure, springel 2005 eg?}

\section{Some Observational Constraints on Galaxy Evolution}
\subsection{Quenching}

An empirical description of galaxy evolution emerges from surveys of galaxy populations and clusters to high redshift.
First among the salient facts is that galaxies form a bimodal distribution in rest-frame color at $z < 2$ \citep{Strateva:2001aa,Baldry:2004aa,Bell:2004aa,Williams:2009tt}, meaning galaxies can be broadly categorized as either actively star-forming spirals (the ``blue cloud"), or quiescent ellipticals and lenticulars (the ``red-sequence").
Although these populations are roughly equivalent in total stellar mass at $z\sim1$, the quiescent galaxy population has nearly doubled in stellar mass, stellar mass density, and number density over the past $\sim7$ Gyr \citep{Arnouts:2007aa,Bell:2004aa,Borch:2006aa,Bundy:2006aa,Brown:2007aa,Faber:2007aa}.
The favored explanation for this is that star-forming galaxies transform into passive ones through a process of ``quenching" \citep{Blanton:2006aa,Bundy:2006aa,Faber:2007aa,Brammer:2011aa}.

Much work is now focused on understanding the specifics of quenching: where and when it happens in the course of a galaxy's life, and how it is physically accomplished.
Quenching is at least partly driven by environmental factors.
A variety of studies at intermediate redshift show that galaxy properties correlate with local environment \citep{Cooper:2006aa,Cooper:2007aa,Quadri:2007aa,Patel:2009aa}, such that groups and clusters contain more quiescent than active galaxies above a given stellar mass \citep{George:2011aa,Muzzin:2012dw,Presotto:2012aa,Tanaka:2012aa,Nantais:2017aa}.
Moreover, with increasing cluster-centric radius, observations find a relative reduction in the number of quiescent systems \cite[e.g.][]{Presotto:2012aa}.
Processes such as galaxy–galaxy mergers \citep{Lavery:1988aa}, harassment and tidal interactions \citep{Moore:1998aa,Bekki:2011aa}, strangulation \citep{Larson:1980aa}, and ram pressure stripping \citep{Gunn:1972aa} are all likely to take place in dense cluster environments, and are possible candidates for environmental quenching mechanisms.

Quenching also correlates with galaxy mass, in that more massive galaxies are more frequently quiescent \citep{Kauffmann:2004aa,Baldry:2006aa,Peng:2010aa}.
Processes intrinsic to the galaxy, such as AGN feedback, or the exhaustion of gas reservoirs by star-formation-driven gas-dynamical outflows are possible candidates for mass-driven quenching mechanisms \citep{McGee:2014aa,Balogh:2016aa}.
%AGN (Benson et al. 2003; Bower et al. 2006; Croton et al. 2006; Menci et al. 2006; Somerville et al. 2008; Schaye et al. 2010; Vogelsberger et al. 2014; Schaye et al. 2015).

Recent simulations have difficulty matching the clustering and abundance of red galaxies, frequently predicting too many quiescent galaxies \citep{Coil:2008aa,Weinmann:2011aa,Vulcani:2014aa}.
This is likely a problem with the given quenching prescription, which has been the subject of some scrutiny \citep{McGee:2009aa,McGee:2011aa,Balogh:2016aa}.
Whatever the mechanism, it must shut off star formation, possibly while effecting morphological transformation \citep{Dressler:1980aa}.
Observations in this area have the potential to constrain the relative contributions of the various nonlinear dynamical and baryonic processes that are thought to occur in dark matter halos.
% understanding it or constraining it will aid our understanding of the processes that shape baryons within DM halos.

\subsection{The Color-Magnitude Relation}

The quiescent population of galaxies exhibits a clear relationship between color and magnitude, such that brighter galaxies are redder \citep{Bower:1992mb,van-Dokkum:1998wd,Baldry:2004aa,Bell:2004qe}.
The color-magnitude relation (CMR) exhibits very little scatter, even over a range of eight in magnitude \citep{Baldry:2004aa}.
This remarkable regularity within the quenched population provides many valuable clues to how quiescent galaxies formed and evolved over cosmic time \citep{Bower:1992mb,Peebles:2002aa}.
% this in turn aids our understanding of the subscale baryonic physics

The CMR is commonly parametrized as a linear relationship between a galaxy's color and its magnitude, with a small, normal scatter in color.
A galaxy's luminosity, being the integrated light of its stellar population, is a good proxy for its stellar mass, especially in near-infrared bands where dust attenuation is minimal.
The interpretation of the color of a stellar population is more subtle, as it depends in general on a galaxy's dust content, age, star formation history, and metallicity.

The slope of the CMR is predominantly the result of a mass-metallicity relation, where more massive galaxies are more metal-enriched \citep[e.g.,][]{Bower:1992mb,Kodama:1997rr,Vazdekis:2001aa,2003AJ....125.1866B}.
The mass-metallicity relation has been studied from the local universe to $z\sim3$ \citep{Tremonti:2004aa,Kewley:2008aa,Andrews:2013aa,Erb:2006aa,Maiolino:2008aa,DeGroot:2016aa}.
Such studies can inform our understanding of the history of galaxy star formation, feedback, outflows, and assembly \citep{Finlator:2008aa,Font:2008aa,Peeples:2011aa}.

The zeropoint of the CMR describes the typical color of a quiescent galaxy at a fixed mass, which evolves with redshift in a manner consistent with the passive aging of a stellar population.
The absolute value of the CMR zeropoint can therefore be interpreted to yield the age of the red-sequence population, or a formation redshift.
The intrinsic color scatter of the CMR is commonly interpreted as intrinsic variation in the physical properties of red-sequence galaxies, independent of the dominant mass effect.
From such considerations, it appears that roughly half of present-day quiescent galaxies were assembled relatively quickly well before $z=1$, and evolved passively to the present day \citep[e.g.,][]{Bower:1998cr,Peebles:2002aa,2003ApJ...596L.143B,Mei:2009wt,Foltz:2015aa}.

The CMR also provides a sensitive diagnostic for simulations that seek to reproduce the properties and distribution of the quiescent population.
Early attempts at modeling the build-up of the red-sequence failed to reproduce the redshift evolution of the CMR slope \citep{Romeo:2008aa,Menci:2008aa}.
In general, reproducing the correct color-magnitude relation still requires some post-processing of the outputs of semi-analytical models \citep[see e.g.][]{Ascaso:2015aa}.
Studying the red-sequence, especially at high redshift, is therefore a basic way to inform our understanding of many aspects of galaxy formation and evolution.
% and thereby complete our picture of baryonic processes

\section{Overview of Studies}

In this work, we present studies of galaxies in two samples of clusters, with the aim of better understanding the history of the quiescent population.

In Chapter \ref{chap-2}, we study the galaxy color-magnitude relation in a sample of galaxy clusters at $z\sim1$.
We use the CMR zeropoint and scatter to constrain the formation redshift for the red-sequence galaxies.
We also compare properties of the CMR as a function of cluster selection method, looking for any differences in the red-sequence as a function of the host halo.

In Chapter \ref{chap-3}, we extend our analysis to a sample of four galaxy clusters at $1.3 < z < 1.65$, using a simple model of quenching to estimate the quenching timescale.
Combining our results with other studies spanning $0 < z < 1.7$, we can compare the redshift evolution of the quenching timescale to that of several relevant timescales.
