\newcommand\mynote[1]{\textcolor{red}{#1}}
\newcommand\persnote[1]{\textcolor{green}{#1}}

\chapter{Introduction}

In recent years, the $\Lambda$CDM paradigm has emerged as the accepted description of the universe.
In this model, dark matter (DM) haloes evolve through hierarchical assembly, being built-up from successive mergers of smaller haloes over cosmological times. \citep{Gott:1975ab,Press:1974aa,White:1978aa,Blumenthal:1984aa,Davis:1985aa,White:1991aa,Barnes:1992aa,Cole:2000aa}.
The dynamical aspects of structure formation in a $\Lambda$CDM universe are well described by the purely gravitational collapse of dissipationless dark matter, and can be studied in large N-body simulations \citep[e.g.,][]{Springel:2005aa,Thomas:1992aa,Navarro:1995aa,Weinberg:2004aa}.
Baryonic matter traces the DM distribution, forming the visible components of these structures.

This $\Lambda$CDM framework has enjoyed dramatic successes in accounting for the observed state of the universe, from the abundance and spatial distribution of the large-scale structures of the universe (clusters, voids, and filaments), to their emergence from the extremely smooth initial conditions indicated by the cosmic microwave background (CMB) \citep{Springel:2006aa}, as well as the temperature anisotropy of the CMB \citep[][and references therein]{Narlikar:2001aa}.

% semianalytic / hydrodynamical simulations?

On smaller, galactic scales (100 kpc), however, the consequences of this model are complicated by the dissipative physics of baryons:
radiative cooling of gas, star formation, active galactic nucleus (AGN) feedback, and chemical enrichment and energy release from supernovae (SN) are poorly\-understood processes, which are currently parameterized by ad hoc recipes in galaxy formation models.
Consequently, the current generation of models have lacked a complete accounting of the build-up of stellar mass over cosmic time.
Better observational constraints on the formation and evolution of galaxies are vitally important in order to complete the description of the relevant physics, from large-scale cosmology to small-scale astrophysics.

\section{Some Observational Constraints on Galaxy Evolution}
\subsection{Quenching}

An empirical description of galaxy evolution emerges from surveys of galaxy populations and clusters to high redshift.
First among the salient facts is that galaxies form a bimodal distribution in rest-frame color at $z < 2$ \citep{Strateva:2001aa,Baldry:2004aa,Bell:2004aa,Williams:2009tt}, meaning galaxies can be broadly categorized as either actively star-forming spirals (the ``blue cloud"), or quiescent ellipticals and lenticulars (the ``red-sequence").
Although these populations are roughly equivalent in total stellar mass at $z\sim1$, the quiescent galaxy population has nearly doubled in stellar mass, stellar mass density, and number density over the past $\sim7$ Gyr \citep{Arnouts:2007aa,Bell:2004aa,Borch:2006aa,Bundy:2006aa,Brown:2007aa,Faber:2007aa}.
The favored explanation for this is that star-forming galaxies transform into passive ones through a process of ``quenching" \citep{Blanton:2006aa,Bundy:2006aa,Faber:2007aa,Brammer:2011aa}.

Much work is now focused on understanding the specifics of quenching: where and when it happens in the course of a galaxy's life, and how it is physically accomplished.
Whatever the mechanism, it must shut off star formation, possibly while effecting morphological transformation \citep{Dressler:1980aa}.
Observations in this area have the potential to constrain the relative contributions of the various nonlinear dynamical and baryonic processes that are thought to occur in dark matter halos.

Quenching is at least partly driven by environmental factors.
A variety of studies at intermediate redshift show that galaxy properties correlate with local environment \citep{Cooper:2006aa,Cooper:2007aa,Quadri:2007aa,Patel:2009aa}, such that groups and clusters contain more quiescent than active galaxies at a given stellar mass \citep{George:2011aa,Muzzin:2012dw,Presotto:2012aa,Tanaka:2012aa,Nantais:2017aa}.
Moreover, with increasing cluster-centric radius, observations find a relative reduction in the number of quiescent systems \cite[e.g.][]{Presotto:2012aa}.
Processes such as galaxy–galaxy mergers \citep{Lavery:1988aa}, harassment and tidal interactions \citep{Moore:1998aa,Bekki:2011aa}, strangulation \citep{Larson:1980aa}, and ram pressure stripping \citep{Gunn:1972aa} are all likely to take place in dense cluster environments, and are possible candidates for environmental quenching mechanisms.

Quenching also correlates with galaxy mass, in that more massive galaxies are more frequently quiescent, independent of the environmental effect \citep{Kauffmann:2004aa,Baldry:2006aa,Peng:2010aa}.
Processes intrinsic to the galaxy, such as AGN feedback, or the exhaustion of gas reservoirs by star-formation-driven gas-dynamical outflows are possible candidates for mass-driven quenching mechanisms \citep{McGee:2014aa,Balogh:2016aa}.

The specific physical mechanisms responsible for quenching remain unclear.
% understanding it or constraining it will aid our understanding of the processes that shape baryons within DM halos.

\subsection{The Color-Magnitude Relation}

The quiescent population of galaxies exhibits a clear relationship between color and magnitude, such that brighter galaxies are redder \citep{Bower:1992mb,van-Dokkum:1998wd,Baldry:2004oq,Bell:2004qe}.
This relation exhibits very little scatter, even over a range of 8 magnitudes \citep{Baldry:2004aa}.
This remarkable regularity within the quenched population contains valuable clues to how quiescent galaxies formed and evolved over cosmic time \citep{Bower:1992mb,Peebles:2002aa}.
% this in turn aids our understanding of the subscale baryonic physics

The color-magnitude relation (CMR) is commonly parametrized as a linear relationship between a galaxy's color and its magnitude, with an additional, small Gaussian scatter in color.
A galaxy's luminosity, being the integrated light of its stellar population, is a good proxy for its stellar mass, especially in near-infrared bands where dust attenuation is minimal.
The interpretation of the color of a stellar population is more subtle, as it depends in general on a galaxy's dust content, age, star formation history, and metallicity.

\citet{Kodama:1997rr} have shown that the slope of the CMR is predominantly a mass-metallicity relation rather than a mass-age relation, due to its lack of evolution with redshift.
Subsequently, the mass-metallicity relation has been studied from the local universe to $z\sim()$ (Tremonti 2004, Kewley & Ellison 2008, Andrews & Martini 2013, Erb 2006, Maiolino 2008).
Such studies can inform our understanding of the history of galaxy star formation, feedback, and outflows (Finlator & dave 2008, peeples & shankdar 2011).

The zeropoint of the CMR describes the typical redness of a quiescent galaxy at a fixed mass, which is seen to evolve with redshift in a manner consistent with the passive evolution of a stellar population.
The absolute value of the CMR zeropoint can therefore be interpreted to yield the age of the red-sequence population, or a formation redshift.
From such considerations, it became quickly apparent that a majority of quiescent galaxies were assembled well before $z=1$
\citep{Peebles:2002aa}
% which has implications for (early quenching of massive halos bower 06)

The intrinsic scatter of the CMR is commonly interpreted as intrinsic variation in the properties of red-sequence galaxies, independent of the dominant mass effect.
This value can constrain the possible variation in galaxy ages, or various star formation scenarios (Bower 1992?).

% The CMR also provides a valuable diagnostic to the semi-analytical and hydrodynamical simulations that seek to reproduce

% taken together, quiescent galaxies formed over a short time when the universe was young, the RS has built up over cosmic time due to quenching.

the study and interpretation of the RS, esp. to high redshift, is a basic way to inform our understanding of galaxy formation and evolution
% and thereby complete our picture of baryonic processes

% open questions: is it present in the blue cloud?
% how is it affected by galaxies migranting onto RS?
% can models reproduce its properties?


\section{Overview of Studies}

In this work, we present studies of the galaxy populations in two samples of clusters with the aim of better understanding the history of quiescent galaxies in the universe.

something more general: red population as function of selection method
measure red-sequence color-mag relation

introduce samples?

