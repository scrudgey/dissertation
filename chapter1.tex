\chapter{Introduction}

In recent years, the $\lambda$CDM paradigm has emerged as the accepted description of the universe.
In this model, dark matter (DM) haloes evolve through hierarchical assembly, being built-up from successive mergers of smaller haloes over cosmological times. (cites)
The dynamical aspects of structure formation in a $\lambda$CDM universe are well described by the purely gravitational collapse of dissipationless dark matter, and can be studied in large N-body simulations (e.g. Springel et al. 2005).
This $\lambda$CDM framework has enjoyed dramatic success in accounting for the observed state of the universe, from the abundance and spatial distribution of the large-scale structures of the universe (clusters, voids, and filaments), to their emergence from the extremely smooth initial conditions indicated by the cosmic microwave background (CMB) (Springel 2006), as well as the temperature anisotropy of the CMB.

On smaller, galactic scales (100 kpc), however, this picture is complicated by the dissipative physics of baryons:
radiative cooling of gas, star formation, active galactic nucleus (AGN) feedback, and chemical enrichment and energy release from supernovae (SN) are poorly—understood processes, which are currently parameterized with ad hoc recipes in galaxy formation models with varying degrees of success.
Consequently, the current generation of models have lacked a complete accounting of galaxy formation and evolution.
A full description of the relevant physics is vitally important to complete the picture from large-scale cosmology to small-scale astrophysics.

An empirical description of galaxy evolution emerges from observations revealed by surveys of galaxy populations and clusters to high redshift.
First among the salient facts revealed by galaxy surveys is that galaxies form a bimodal distribution in rest-frame color at $z < 2$ \citep{Strateva:2001aa,Baldry:2004aa,Bell:2004aa,Williams:2009tt}, meaning galaxies can be broadly categorized as either actively star-forming spirals (the ``blue cloud"), or quiescent ellipticals and lenticulars (the ``red-sequence").
Although these populations are roughly equivalent in total stellar mass at $z\sim1$, the quiescent galaxy population has nearly doubled in stellar mass, stellar mass density, and number density over the past $\sim7$ Gyr \citep{Arnouts:2007aa,Bell:2004aa,Borch:2006aa,Bundy:2006aa,Brown:2007aa,Faber:2007aa}.
The favored explanation for this is that active galaxies transform into passive ones through a process of ``quenching". (Blanton 2006; Bundy et al. 2006; Faber et al. 2007; Brammer et al. 2011)

Although much evidence exists for quenching driven both by galaxy stellar mass and environment, the specific physical mechanisms responsible remain unclear.
Processes such as galaxy–galaxy mergers (Lavery & Henry 1988), harassment and tidal interactions (Moore, Lake & Katz 1998; Bekki & Couch 2011), strangulation (Larson, Tinsley & Caldwell 1980), and ram pressure stripping (Gunn & Gott 1972) are all likely to take place in the dense environments of clusters and groups of galaxies.
At the same time, processes intrinsic to the galaxy, such as AGN feedback, or the exhaustion of gas reservoirs by star-formation-driven gas-dynamical outflows have also been proposed as quenching mechanisms \citep{McGee:2014aa,Balogh:2016aa}.

motivates paper 1, paper 2

introduce samples?

