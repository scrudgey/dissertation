\chapter{Summary}

In this work we have presented two studies of galaxy populations in cluster samples at $z\sim1$ and $z\sim1.5$ with the aim of constraining galaxy formation and evolutionary processes.
In Chapter \ref{chap-2} we studied the red-sequence in a sample of ten galaxy clusters at $z\sim1$.
In Chapter \ref{chap-3} we modeled the quenching process in the same sample of clusters as well as a higher-redshift sample of four clusters at $z\sim1.5$.

\section{The Color-Magnitude Relation at $z\sim1$}

The color-magnitude relation can be modeled by a metallicity sequence of simple stellar populations formed at high redshift.
After normalizing the magnitude of these model SSPs using the CMR in the local universe, the evolution of the CMR can be modeled, allowing for a physical interpretation of its slope, scatter, and intercept.

Starting with the above considerations, we measure the CMR in the sample of ten GCLASS clusters at $z\sim1$.
Extensive spectroscopy combined with interpolated rest-frame colors allows for a precise determination of the CMR.

The CMR zeropoint is not seen to evolve with redshift to $z=1.3$.
From comparison with models, this indicates that the quiescent population in these clusters is mature, and formed at $z_f \geq 3$.

Interpretation of the CMR intrinsic scatter is more complicated, and may be the result of several factors, including galaxy ages and metallicities.
If we assume the intrinsic scatter is due solely to a scatter in ages, then it can constrain an upper-bound age spread.
For galaxies with $z_f=3$, we find the scale of the intrinsic scatter is consistent with an average age spread of 1 Gyr.
\mynote{actual values?}

% comparison with previous study findings?

We compare our results with those derived for a sample of X-ray-selected clusters at a similar redshift.
X-ray- or red-sequence-selected clusters are well-known to be inherently biased toward either gas-rich systems or galaxy-rich systems.
However, within uncertainties, we find no difference in the red-sequence slope, scatter, or intercept between the two cluster samples.
The galaxy populations within these cluster samples are insensitive to any cluster biases that may be present.
This result provides confidence in studies of cluster galaxy populations that drawn on results obtained in clusters selected by different methods.

so, no evidence for ``halo bias"

We compare the quiescent galaxy populations identified by spectroscopic features versus rest-frame \textit{UVJ} color-color selection.
14\% of the \textit{UVJ}-quiescent population show [O\textsc{II}] emission, while 16\% of the UVJ-star-forming galaxies exhibit no [O\textsc{II}] emission.

\section{Evolution in Quenching Time from $z=0$ to $z=1.5$}

We present a toy model of environmental quenching that describes the build-up of the quenched and star-forming galaxy populations in terms of a cluster mass accretion rate and a quenching time $t_Q$.
In this model, star-forming galaxies that are accreted by the cluster remain star-forming for a delay time, $t_D$, after which their star formation rate rapidly decreases to zero over a fade time, $t_F$.
With simulated cluster halo mass accretion rates taken from the Millennium-\textsc{II} simulation, the numbers of observed star-forming, intermediate, and quenched galaxies can then constrain the fade and delay times, and thereby the total quench time.
We apply this modeling to the $z\sim1$ and $z\sim1.5$ cluster samples, finding a $t_Q$ of $\simgresult$ and $\simhiresult$, respectively.

Assembling measurements of $t_Q$ found in the literature, we find that the quenching time has decreased significantly with redshift.
While in the local universe $t_Q$ is on the order of 4-5 Gyr, it has decreased to the order of $\sim1$ Gyr at $z\sim1.5$.

The time evolution of $t_Q$ can be compared with the expected evolution of various characteristic timescales.
We compare the redshift behavior of $t_Q$ with the modeled time evolution of the total gas depletion time, global star-formation rates, an SFR-outflow model, and the dynamical (crossing) time, $t_\mathrm{dyn}$.
The quenching time is seen to evolve faster than the gas depletion timescale and slower than the SFR-outflow timescale, but is consistent with the evolution of the dynamical time.

Additionally, $t_Q$ exhibits dependence on the mass of the host halo, with $t_Q$ in groups being longer by $\sim2$ Gyr than in clusters, at low redshift.
The group quenching time is also seen to evolve as the dynamical time, while remaining longer than the cluster quenching time at all redshifts.

The dynamical time is a function of the global properties of the host cluster (velocity dispersions, halo masses), and not related to the properties of the galaxy (star formation rate, gas fraction, stellar mass).
The above results both suggest that galaxies' quenching time depends on the dynamical aspects of their host halos.
This would be the case for any quenching mechanism that relies on the galaxy making one or more passes through a certain radius of the host cluster halo, such as ram-pressure-based quenching scenarios.
This result suggests that processes intrinsic to the galaxy itself, such as

\section{Future Work}

The suggestion that quenching could be driven by ram-pressure stripping is perhaps surprising given the less-evolved nature of clusters and their gas component at high redshift.
Searching for more evidence of ram-pressure stripping to support this scenario is a possible avenue to pursue.

The major results of Chapter \ref{chap-3} depend on comparisons of $t_Q$ measured by multiple teams with multiple different methods.
However, the model we present is fairly simple, relying simply on number counts of galaxies in different evolutionary phases.
A large, homogenous sample of cluster galaxies studied over a wide range of redshifts would eliminate bias effects.

The $t_Q$ model of Chapter \ref{chap-3} describes the redshift evolution of the passive and intermediate populations.
It should therefore be possible to make contact with the observed evolution in the cluster passive fraction over cosmological time, as a further test of the model.

The $t_Q$ model also assumes a fixed value for $t_Q$, which is not the case.
Over the redshift range probed in this work, $t_Q$ is expected to be short and evolve slowly, meaning that differences due to time-averaging $t_Q$ ought to be small.
A complete model of the population dynamics within clusters should allow for $t_Q$ to be a general function of cosmic time.


simulations (see adam comments)
