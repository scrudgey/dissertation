\chapter{Summary}

In this work we have presented two studies of galaxy populations in cluster samples at $z\sim1$ and $z\sim1.5$ with the aim of constraining galaxy formation and evolutionary processes.
In Chapter \ref{chap-2} we studied the red-sequence in a sample of ten galaxy clusters at $z\sim1$.
In Chapter \ref{chap-3} we modeled the quenching process for the same sample of clusters as well as a sample of four clusters at $z\sim1.5$.

\section{The Color-Magnitude Relation at $z\sim1$}

The linear form of the color-magnitude relation can be modeled by a metallicity sequence of simple stellar populations formed at high redshift, normalized to reproduce the CMR in the local universe.
The evolution of this model CMR allows for a physical interpretation of its slope, scatter, and intercept.
Starting from these considerations, we study the CMR in the GCLASS sample of ten red-sequence-selected clusters at $z\sim1$.
Extensive spectroscopy, combined with interpolated rest-frame colors, allows for a precise determination of the CMR in these clusters.

The CMR zeropoint does not evolve with redshift to $z=1.3$.
From comparison with models, this indicates that the quiescent population in these clusters is mature, and formed the bulk of its stars at $z_f \geq 3$.
Interpretation of the CMR intrinsic scatter is more complicated, as it may be the result of several factors, including variation in galaxy ages and metallicities.
If we assume the intrinsic scatter is due solely to a scatter in ages, then it can constrain the spread in ages for these galaxies.
For galaxies with $z_f=3$, we find the scale of the intrinsic scatter is consistent with an average age spread of $\Delta t \geq 1$ Gyr.

% comparison with previous study findings?

We compare our results with an X-ray-selected sample of clusters at a similar redshift.
X-ray- or red-sequence-selected clusters are known to be inherently biased toward either gas-rich systems or galaxy-rich systems.
However, within uncertainties, we find no difference in the red-sequence slope, scatter, or intercept between the two cluster samples.
The quiescent galaxy populations within these cluster samples are insensitive to any cluster biases that may be present.
% This result provides confidence in studies of cluster galaxy populations that drawn on results obtained in clusters selected by different methods.

We compare the quiescent galaxy populations identified by spectroscopic features versus rest-frame \textit{UVJ} color-color selection.
14\% of the \textit{UVJ}-quiescent population show [O\textsc{II}] emission, while 16\% of the UVJ-star-forming galaxies exhibit no [O\textsc{II}] emission.
% This is likely due to the empirical nature of the \textit{UVJ}

\section{Evolution in Quenching Time from $z=0$ to $z=1.5$}

We present a toy model of environmental quenching that describes the build-up of the quenched and star-forming galaxy populations in terms of a cluster mass accretion rate and a quenching time $t_Q$.
In this model, star-forming galaxies that are accreted by the cluster remain star-forming for a delay time, $t_D$, after which their star formation rate decreases to zero over a fade time, $t_F$.
With cluster halo mass accretion rates taken from the Millennium-\textsc{II} simulation, the numbers of observed star-forming, intermediate, and quenched galaxies can constrain the fade and delay times, and thereby the total quench time.
We apply this modeling to intermediate-mass galaxies ($M_* \geq 10^{10.5}~\mathrm{M}_\odot$) in the $z\sim1$ and $z\sim1.5$ cluster samples, finding a $t_Q$ of $\simgresult$ Gyr and $\simhiresult$ Gyr, respectively.
% These values are for galaxies of mass $M_* \geq 10^{10.5}~\mathrm{M}_\odot$.

Assembling measurements of $t_Q$ found in the literature, we find that the quenching time has decreased significantly with redshift.
While in the local universe $t_Q$ is roughly 4-5 Gyr, it has decreased to the order of $\sim1$-2 Gyr at $z\sim1.5$.
% The time evolution of $t_Q$ can be compared with the evolution of various characteristic timescales.
We compare the redshift behavior of $t_Q$ with the modeled time evolution of the total gas depletion time, global star-formation rates, an SFR-outflow model, and the dynamical (crossing) time, $t_\mathrm{dyn}$.
The quenching time is seen to evolve faster than the gas depletion timescale and slower than the SFR-outflow timescale, but is consistent with the evolution of the dynamical time.
Additionally, $t_Q$ exhibits dependence on the mass of the host halo: at low redshift, $t_Q$ is longer by $\sim2$ Gyr in groups than in clusters, and the group quenching time remains longer than the cluster quenching time at all redshifts probed.

Care must be taken when interpreting this result.
The dynamical time is a function of the global properties of the host cluster (velocity dispersions, halo masses), not the properties of the galaxy (star formation rate, gas fraction, stellar mass).
That $t_Q$ evolves like $t_\mathrm{dyn}$, and that it depends on host halo mass both suggest that the quenching time of these galaxies is driven by the dynamical properties of their host halos.
This is expected for quenching mechanisms that rely on the galaxy making one or more passes through a certain radius of the host cluster halo, such as ram-pressure or tidal stripping scenarios.
However, it is also possible that environment affects the efficiency of mass-driven quenching mechanisms, as has been suggested by \citet{Henriques:2017aa}.
We caution that it may be the case that the relevant physics do not separate cleanly into environmental and intrinsic mechanisms, but instead result from an interplay between the properties of a galaxy and its host halo.
% Wang & White (2012) showed that this effect is indeed present in the Guo et al. (2011)

\section{Future Work}

% The question of what mechanisms are responsible for quenching is currently the subject of active research.
% The issue is complicated by the fact that quenching is likely to be influenced by many factors, including redshift, environment, and galaxy mass.
% In fact, it has recently been suggested that the efficiency of environmental- and mass-driven quenching effects are interrelated, especially for mass-quenching driven by AGN feedback \citep{Henriques:2017aa}.
% Searching for more observational evidence (a ``smoking gun") of environmental-driven effects to support the conclusions of Chapter \ref{chap-3} is one avenue to pursue.

The major results of Chapter \ref{chap-3} depend on comparisons drawn between values of $t_Q$ reported in the literature.
The various studies included on Figure \ref{fig-tq} generally differ in their datasets, their definitions of star-forming and quenched galaxies, and their methods of associating timescales with the properties of galaxy populations.
A stronger result would be obtained if our quenching-time model could be applied to a single, homogeneously-selected sample of clusters and galaxies spanning $0 < z < 1.6$.
Stacking results over a larger cluster sample also serves to reduce uncertainty due to cosmic variance and differences in cluster accretion histories.

% The $t_Q$ model of Chapter \ref{chap-3} makes predictions for the redshift evolution of the passive and intermediate populations.
% Knowing how $t_Q$ evolves with redshift should make it possible for the model to make contact with the observed evolution in cluster passive fractions over cosmological time.
%these are not independent.

The $t_Q$ model as presented assumes a fixed value for $t_Q$.
This means that the value measured for $t_Q$ is time-averaged over the past history of the cluster, weighted by the mass accretion rate.
As the quenching time is short and evolves slowly over $1.0 < z < 1.6$, while the mass accretion rate increases with time, this assumption does not greatly influence our results.
Because most of the mass of a cluster has been accreted in its recent past, the time-averaged $t_Q$ is closer to the instantaneous value.
This effect is not present in the studies assembled on Figure \ref{fig-tq} that measure $t_Q$ from the properties of the star-forming population, or from a short-lived quenched population, and the overall dynamical evolutionary trend of $t_Q$ should therefore not be affected by this consideration.
Nevertheless, it should be possible to modify the RGB model to account for a time-varying $t_Q(z)$.
This modification would be necessary when applying the model to lower-redshift clusters where a history of variation in $t_Q(z)$ has a much larger effect on the quenched fraction.

More precise uncertainties on the values of $t_D$, $t_F$, and $t_Q$ could be obtained by the inclusion of

The results of Chapter \ref{chap-3} suggest that ram-pressure or tidal stripping should not be discounted as possible quenching mechanisms.
Models of stripping that take into account galaxy mass, orbit, structural parameters, and stripping rate have been shown to provide much better fits to observed galaxy properties than early, crude models which assumed instantaneous stripping \citep{McCarthy:2008aa,Font:2008ab,Henriques:2017aa}.
It is possible that more development in this area could improve the results of future semi-analytic models.

% simulations (see adam comments)
