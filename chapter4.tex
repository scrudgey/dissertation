\chapter{Summary}

In this work we have presented two studies of galaxy populations in cluster samples at $z\sim1$ and $z\sim1.5$ with the aim of constraining galaxy formation and evolutionary processes.
In Chapter \ref{chap-2} we studied the red-sequence in a sample of ten galaxy clusters at $z\sim1$.
In Chapter \ref{chap-3} we modeled the quenching process in the same sample of clusters as well as a higher-redshift sample of four clusters at $z\sim1.5$.

\section{The Color-Magnitude Relation at $z\sim1$}

* CMR analysis results
  * zeropoint
  * scatter
  * slope?

comparison with previous studies that have found this?

no apparent cluster selection effects at $z=1$.
perhaps surprising agreement in zeropoint considering
so, no evidence for ``halo bias"

comparison with UVJ selection

effectiveness of UVJ selection

\section{Evolution in Quenching Time from $z=0$ to $z=1.5$}

describe toy model and technique

tQ results in cluster samples

tQ seems to evolve as the dynamical time

comparison with model timescales

\section{Future Work}

larger, uniform sample from z=0 to 1.6 to interpret changes in cluster population

comparing with observed passive fractions

unpacking the time-averaged tq

simulations (see adam comments)
